\documentclass{article}

\usepackage{pbox}
\usepackage{graphicx}
\usepackage{float}
\usepackage{fancyvrb}
\usepackage[T1]{fontenc}
\usepackage{lmodern}
\usepackage{setspace}
\usepackage[nottoc]{tocbibind}
\usepackage[font=large]{caption}
\usepackage{framed}
\usepackage{tabularx}
\usepackage{amsmath}
\usepackage{hyperref}
\usepackage{fontspec}
\usepackage[backend=biber,sorting=none]{biblatex}
%%\usepackage[
%%	backend=biber,
%%	style=ieee,
%%	sorting=none
%%]{biblatex}
%\addbibresource{project_refs.bib}

%% Hide section, subsection, and subsubsection numbering
%\renewcommand{\thesection}{}
%\renewcommand{\thesubsection}{}
%\renewcommand{\thesubsubsection}{}

% Alternative form of doing section stuff
\renewcommand{\thesection}{}
\renewcommand{\thesubsection}{\arabic{section}.\arabic{subsection}}
\makeatletter
\def\@seccntformat#1{\csname #1ignore\expandafter\endcsname\csname the#1\endcsname\quad}
\let\sectionignore\@gobbletwo
\let\latex@numberline\numberline
\def\numberline#1{\if\relax#1\relax\else\latex@numberline{#1}\fi}
\makeatother

\makeatletter
\renewcommand\tableofcontents{%
    \@starttoc{toc}%
}
\makeatother

\newcommand{\respacing}{\doublespacing \singlespacing}
\newcommand{\code}[2][1]{\noindent \texttt{\tab[#1] #2} \\}
\newcommand{\skipline}[0]{\texttt{}\\}
\newcommand{\tnl}[0]{\tabularnewline}

\begin{document}
%--------
	\font\titlefont={Times New Roman} at 20pt
	\title{{\titlefont Volt32 CPU}}

	\font\bottomtextfont={Times New Roman} at 12pt
	\date{{\bottomtextfont} \today}
	\author{{\bottomtextfont Andrew Clark}}

	\setmainfont{Times New Roman}
	\setmonofont{Courier New}

	\maketitle
	\pagenumbering{gobble}

	\newpage
	\pagenumbering{arabic}
	%\tableofcontents
	%\newpage

	%\doublespacing

%\section{Abstract}
	%\setcounter{section}{-1}

\section{Table of Contents}
	\tableofcontents
	\newpage

\section{Registers, Main Widths, etc.}
	\begin{itemize}
	%--------
	\item The main width of the processor is 32-bit.  Addresses are 32-bit,
	and some operations forcibly treat the operands as 32-bit.

	\item The machine is an implementation of Line Associative Registers
	(LARs).  Both instruction LARs (ILARs) and data LARs (DLARs) are
	included in the design.  There are a grand total of 128 ILARs and 128
	DLARs, but they are split between the LARs owned by the supervisor
	mode and the LARs owned by the user mode.  There are 64 supervisor mode
	ILARs and 64 supervisor mode DLARs, and there are 64 user mode ILARs
	and 64 user mode DLARs.

	\item ILARs
		\begin{itemize}
		%--------
		\item In user mode, ILARs 0 to 63 are referred to as \texttt{i0},
		\texttt{i1}, \texttt{i2}, ..., \texttt{i61}, \texttt{i62},
		\texttt{i63}.

		\item In supervisor mode, ILARs 64 to 127 are referred to as
		\texttt{i0}, \texttt{i1}, \texttt{i2}, ..., \texttt{i61},
		\texttt{i62}, \texttt{i63}.

		\item The two ILARs with the name \texttt{i0} have all their fields
		set to zero, and written to, the contents do not change.

		\item An ILAR's data field is 128 bytes long.  It is composed of
		32-bit instructions aligned to 32 bits.

		\item An ILAR's scalar offset field is 5-bit due to instructions
		being 32-bit and the data field being 128 bytes long.

		\item The base address field of an ILAR is 32 - 7 = 25-bit.
		%--------
		\end{itemize}

	\item DLARs
		\begin{itemize}
		%--------
		\item In user mode, DLARs 0 to 63 are referred to as \texttt{d0},
		\texttt{d1}, \texttt{d2}, ..., \texttt{d61}, \texttt{d62},
		\texttt{d63}

		\item In supervisor mode, DLARs 0 to 63 are referred to as
		\texttt{d0}, \texttt{d1}, \texttt{d2}, ..., \texttt{d61},
		\texttt{dira}, \texttt{dida}.

		\item The two DLARs with the name \texttt{d0} have all their fields
		set to zero, and written to, the contents do not change.

		%--------
		\end{itemize}
	%--------
	\end{itemize}


	%\printbibliography[heading=bibnumbered,title={Bibliography}]

%--------
\end{document}
