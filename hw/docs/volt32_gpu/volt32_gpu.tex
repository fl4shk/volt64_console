\documentclass{article}

\usepackage{pbox}
\usepackage{graphicx}
\usepackage{float}
\usepackage{fancyvrb}
\usepackage[T1]{fontenc}
\usepackage{lmodern}
\usepackage{setspace}
\usepackage[nottoc]{tocbibind}
\usepackage[font=large]{caption}
\usepackage{framed}
\usepackage{tabularx}
\usepackage{amsmath}
\usepackage{hyperref}
\usepackage{fontspec}
\usepackage[backend=biber,sorting=none]{biblatex}
%%\usepackage[
%%	backend=biber,
%%	style=ieee,
%%	sorting=none
%%]{biblatex}
%\addbibresource{project_refs.bib}

%% Hide section, subsection, and subsubsection numbering
%\renewcommand{\thesection}{}
%\renewcommand{\thesubsection}{}
%\renewcommand{\thesubsubsection}{}

% Alternative form of doing section stuff
\renewcommand{\thesection}{}
\renewcommand{\thesubsection}{\arabic{section}.\arabic{subsection}}
\makeatletter
\def\@seccntformat#1{\csname #1ignore\expandafter\endcsname\csname the#1\endcsname\quad}
\let\sectionignore\@gobbletwo
\let\latex@numberline\numberline
\def\numberline#1{\if\relax#1\relax\else\latex@numberline{#1}\fi}
\makeatother

\makeatletter
\renewcommand\tableofcontents{%
    \@starttoc{toc}%
}
\makeatother

\newcommand{\respacing}{\doublespacing \singlespacing}
\newcommand{\code}[2][1]{\noindent \texttt{\tab[#1] #2} \\}
\newcommand{\skipline}[0]{\texttt{}\\}
\newcommand{\tnl}[0]{\tabularnewline}

\begin{document}
%--------
	\font\titlefont={Times New Roman} at 20pt
	\title{{\titlefont Volt32 GPU}}

	\font\bottomtextfont={Times New Roman} at 12pt
	\date{{\bottomtextfont} \today}
	\author{{\bottomtextfont Andrew Clark}}

	\setmainfont{Times New Roman}
	\setmonofont{Courier New}

	\maketitle
	\pagenumbering{gobble}

	\newpage
	\pagenumbering{arabic}
	%\tableofcontents
	%\newpage

	%\doublespacing

%\section{Abstract}
	%\setcounter{section}{-1}

\section{Table of Contents}
	\tableofcontents
	\newpage

\section{General}
	\begin{itemize}
	%--------
	\item The GPU is a 2D one, with sprites and backgrounds.

	\item Tiles are 8x8.

	\item The screen resolution of the console is 320x240, or 40 tiles by
	30 tiles.

	\item There are 76,800 bytes, or 320 * 240 bytes, of tile VRAM.  This
	is enough for 1,200 tiles, which is a whole BG's worth of tiles.

	\item The GPU can also be set to treat tile VRAM as a single
	framebuffer.
	%--------
	\end{itemize}
	\newpage

\section{Sprites}
	\begin{itemize}
	%--------
	\item There are 128 sprites.

	\item Sprites can be of two different sizes:  8x8 and 16x16.

	\item There can be up to 64 sprites per scanline.

	\item Sprites have a 3-bit drawing priority field.  If two sprites have
	the same drawing priority and collide with one another, the lower
	numbered sprite's pixel will be drawn instead of the one with the higher
	number's pixel.  If a sprite has the same or more priority as a
	background, it will be drawn on top of the background, not counting
	when a sprite will be drawn on top of that sprite.

	\item The format of a sprite attribute table (SAT) entry is as follows:
		\begin{table}[H]
			\begin{center}
				\begin{tabular}{|c|c|c|}
					\hline
					\textbf{Name} & \textbf{Bit Range} 
						& \textbf{Description}\\
					\hline
					Reserved[8] & [31:24]
						& \textit{Reserved for future expansion}\\
					Size[0:0] & [23]
						& \texttt{0b0} for 8x8, \texttt{0b1} for 16x16 \\
					Priority[3] & [22:20]
						& Drawing priority\\
					Tile Index[17] & [19:3]
						& Which tile to draw\\
					Horizontal Flip[1] & [2]
						& Whether or not to flip the sprite horizontally
						when it's drawn\\
					Vertical Flip[1] & [1]
						& Whether or not to flip the sprite vertically
						when it's drawn\\
					Reserved[1] & [0]
						& \textit{Reserved for future expansion}\\
					\hline
				\end{tabular}
			\end{center}
		\end{table}
	%--------
	\end{itemize}
	\newpage

\section{Backgrounds}
	\begin{itemize}
	%--------
	\item There are 4 backgrounds.

	\item Backgrounds have a 3-bit drawing priority field.  In the case of
	a tie in drawing priority between two backgrounds, the lower numbered
	background's pixel will be drawn on top of the higher numbered
	background's pixel.

	\item Backgrounds can be scrolled vertically and horizontally.

	\item Tilemaps are 512x256, or 64 tiles by 32 tiles.  There is a grand
	total of 16 kiB of tilemap space, or 16 bits per every background tile.

	\item The format of a tilemap entry is as follows:
		\begin{table}[H]
			\begin{center}
				\begin{tabular}{|c|c|c|}
					\hline
					\textbf{Name} & \textbf{Bit Range}
						& \textbf{Description}\\
					\hline
					\hline
				\end{tabular}
			\end{center}
		\end{table}
	%--------
	\end{itemize}
	\newpage


	%\printbibliography[heading=bibnumbered,title={Bibliography}]

%--------
\end{document}
