\documentclass{article}

\usepackage{pbox}
\usepackage{graphicx}
\usepackage{float}
\usepackage{fancyvrb}
\usepackage[T1]{fontenc}
\usepackage{lmodern}
\usepackage{setspace}
\usepackage[nottoc]{tocbibind}
\usepackage[font=large]{caption}
\usepackage{framed}
\usepackage{tabularx}
\usepackage{amsmath}
\usepackage{hyperref}
\usepackage{fontspec}
\usepackage[backend=biber,sorting=none]{biblatex}
%%\usepackage[
%%	backend=biber,
%%	style=ieee,
%%	sorting=none
%%]{biblatex}
%\addbibresource{project_refs.bib}

%% Hide section, subsection, and subsubsection numbering
%\renewcommand{\thesection}{}
%\renewcommand{\thesubsection}{}
%\renewcommand{\thesubsubsection}{}

% Alternative form of doing section stuff
\renewcommand{\thesection}{}
\renewcommand{\thesubsection}{\arabic{section}.\arabic{subsection}}
\makeatletter
\def\@seccntformat#1{\csname #1ignore\expandafter\endcsname\csname the#1\endcsname\quad}
\let\sectionignore\@gobbletwo
\let\latex@numberline\numberline
\def\numberline#1{\if\relax#1\relax\else\latex@numberline{#1}\fi}
\makeatother

\makeatletter
\renewcommand\tableofcontents{%
    \@starttoc{toc}%
}
\makeatother

\newcommand{\respacing}{\doublespacing \singlespacing}
\newcommand{\code}[2][1]{\noindent \texttt{\tab[#1] #2} \\}
\newcommand{\skipline}[0]{\texttt{}\\}
\newcommand{\tnl}[0]{\tabularnewline}

\begin{document}
%--------
	\font\titlefont={Times New Roman} at 20pt
	\title{{\titlefont Volt32 GPU}}

	\font\bottomtextfont={Times New Roman} at 12pt
	\date{{\bottomtextfont} \today}
	\author{{\bottomtextfont Andrew Clark}}

	\setmainfont{Times New Roman}
	\setmonofont{Courier New}

	\maketitle
	\pagenumbering{gobble}

	\newpage
	\pagenumbering{arabic}
	%\tableofcontents
	%\newpage

	%\doublespacing

%\section{Abstract}
	%\setcounter{section}{-1}

\section{Table of Contents}
	\tableofcontents
	\newpage

\section{General}
	\begin{itemize}
	%--------
	\item The GPU is a 2D one, with sprites and backgrounds.

	\item Tiles are 8x8.

	\item The screen resolution of the console is 320x240, or 40 tiles by
	30 tiles.

	\item There are 76,800 bytes, or 320 * 240 bytes, of tile VRAM.  This
	is enough for 1,200 tiles, which is a whole BG's worth of tiles.

	\item The GPU can also be set to treat tile VRAM as a single
	framebuffer.
	%--------
	\end{itemize}
	\newpage

\section{Sprites}
	\begin{itemize}
	%--------
	\item There are 128 sprites.

	\item Sprites can be of two different sizes:  8x8 and 16x16.

	\item There can be up to 64 sprites per scanline.

	\item Sprites have a 3-bit drawing priority field.  If two sprites have
	the same drawing priority and collide with one another, the lower
	numbered sprite's pixel will be drawn instead of the one with the higher
	number's pixel.  If a sprite has the same or more priority as a
	background, it will be drawn on top of the background, not counting
	when a sprite will be drawn on top of that sprite.

	\item The format of a sprite attribute table (SAT) entry is as follows:
		\begin{table}[H]
			\begin{center}
				\begin{tabular}{|c|c|p{65mm}|}
					\hline
					\textbf{Name} & \textbf{Bit Range} 
						& \textbf{Description}\\
					\hline
					Reserved[23] & [63:41]
						& \textit{Reserved for future expansion}\\
					Horizontal Position[9] & [40:32]
						& Horizontal position on screen (\texttt{0x0} means
						left edge of screen)\\
					Vertical Position[9] & [31:23]
						& Vertical position on screen (\texttt{0x0} means
						top edge of screen)\\
					Size[0:0] & [22]
						& \texttt{0b0} for 8x8, \texttt{0b1} for 16x16 \\
					Priority[3] & [21:19]
						& Drawing priority\\
					Horizontal Flip[1] & [18]
						& Whether or not to flip the sprite horizontally
						when it's drawn\\
					Vertical Flip[1] & [17]
						& Whether or not to flip the sprite vertically
						when it's drawn\\
					Tile Index[17] & [16:0]
						& Which tile to draw (8x8), or which top-left tile
						to draw (16x16) \\
					\hline
				\end{tabular}
			\end{center}
		\end{table}

	\item 1024 bytes are required to store the SAT.
	%--------
	\end{itemize}
	\newpage

\section{Backgrounds}
	\begin{itemize}
	%--------
	\item There are 4 backgrounds.

	\item Backgrounds have a 3-bit drawing priority field.  In the case of
	a tie in drawing priority between two backgrounds, the lower numbered
	background's pixel will be drawn on top of the higher numbered
	background's pixel.

	\item Backgrounds can be scrolled vertically and horizontally.

	\item Tilemaps are 512x256, or 64 tiles by 32 tiles.  There is a grand
	total of 8 kiB of tilemap space, or 32 bits per every background tile.

	\item The format of a tilemap entry is as follows:
		\begin{table}[H]
			\begin{center}
				\begin{tabular}{|c|c|c|}
					\hline
					\textbf{Name} & \textbf{Bit Range}
						& \textbf{Description}\\
					\hline
					Reserved[10] & [31:22]
						& \textit{Reserved for future expansion}\\
					Priority[3] & [21:19]
						& Drawing priority\\
					Horizontal Flip[1] & [18]
						& Whether or not to flip the tile horizontally
						when it's drawn\\
					Vertical Flip[1] & [17]
						& Whether or not to flip the tile vertically
						when it's drawn\\
					Tile Index[17] & [16:0]
						& Which tile to draw\\
					\hline
				\end{tabular}
			\end{center}
		\end{table}
	%--------
	\end{itemize}
	\newpage

\section{Palette}
	\begin{itemize}
	%--------
	\item There is a single 256 color palette, though color 0 is the
	transparent color.

	\item Colors are 18-bit, with 6-bit channels.

	\item A structure of arrays is used to store the entire palette.  This
	is due to the CPU's registers being vectors. Each array stores one RGB
	channel as an 8-bit field.  Bits 7 and 6 of each channel table (CHT)
	entry are ignored by the GPU.

	\item 768 bytes are needed to store the palette. 

	\item The GPU's framebuffer mode still uses the 256-entry palette, such
	that it is a paletted framebuffer.  For this mode, the GPU treats color
	0 as a visible color instead of transparent.
	%--------
	\end{itemize}
	\newpage

\section{Misc. Registers}
	\begin{itemize}
	%--------
	\item BG0 Horizontal Scroll[32]
	\item BG0 Vertical Scroll[32]

	\item BG1 Horizontal Scroll[32]
	\item BG1 Vertical Scroll[32]

	\item BG2 Horizontal Scroll[32]
	\item BG2 Vertical Scroll[32]

	\item BG3 Horizontal Scroll[32]
	\item BG3 Vertical Scroll[32]

	\item GPU Status[32]
		\begin{table}[H]
			\begin{center}
				\begin{tabular}{|c|c|p{65mm}|}
					\hline
					\textbf{Name} & \textbf{Bit Range}
						& \textbf{Description}\\
					\hline
					Reserved[31] & [31:30]
						& \textit{Reserved for future expansion}\\
					Framebuffer Mode Enable[0] & [0:0]
						& \texttt{0b1} to enable the GPU's framebuffer mode
						(which disables sprites and backgrounds and uses
						tile VRAM as a framebuffer)\\
					\hline
				\end{tabular}
			\end{center}
		\end{table}
	%--------
	\end{itemize}
	\newpage

\section{IO Space Map}
	\begin{itemize}
	%//--------
	\item IO Space is referred to as such because it's IO from the
	perspective of the CPU.  The CPU itself has two address spaces it can
	access:  IO and main memory.  The CPU is the only "bus master" that can
	access main memory.

	\item The "\textbf{Offset}" in the "\textbf{Offset}" and 
	"\textbf{Offset + Size}" columns means "offset from base address",
	where "base address" is the IO space base address of the GPU.

	\item Here is the table proper.
		\begin{table}[H]
			\begin{center}
				\begin{tabular}{|c|c|c|}
					\hline
					\textbf{Offset} & \textbf{Offset + Size - 1}
						& \textbf{Name}\\
					\hline
					\texttt{0x0} & \texttt{0x1\_2bff} & Tiles\\
					\texttt{0x1\_2c00} & \texttt{0x1\_4bff} & Tilemaps\\
					\texttt{0x1\_4c00} & \texttt{0x1\_4fff}
						& Sprite Attribute Table\\
					\texttt{0x1\_5000} & \texttt{0x1\_52ff}
						& Palette\\
					\texttt{0x1\_5300} & \texttt{0x1\_53ff}
						& Misc. Registers\\
					\hline
				\end{tabular}
			\end{center}
		\end{table}
	%--------
	\end{itemize}
	\newpage

	%\printbibliography[heading=bibnumbered,title={Bibliography}]

%--------
\end{document}
