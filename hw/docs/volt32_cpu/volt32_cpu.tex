\documentclass{article}

\usepackage{pbox}
\usepackage{graphicx}
\usepackage{float}
\usepackage{fancyvrb}
\usepackage[T1]{fontenc}
\usepackage{lmodern}
\usepackage{setspace}
\usepackage[nottoc]{tocbibind}
\usepackage[font=large]{caption}
\usepackage{framed}
\usepackage{tabularx}
\usepackage{amsmath}
\usepackage{hyperref}
\usepackage{fontspec}
\usepackage[backend=biber,sorting=none]{biblatex}
%%\usepackage[
%%	backend=biber,
%%	style=ieee,
%%	sorting=none
%%]{biblatex}
%\addbibresource{project_refs.bib}

%% Hide section, subsection, and subsubsection numbering
%\renewcommand{\thesection}{}
%\renewcommand{\thesubsection}{}
%\renewcommand{\thesubsubsection}{}

% Alternative form of doing section stuff
\renewcommand{\thesection}{}
\renewcommand{\thesubsection}{\arabic{section}.\arabic{subsection}}
\makeatletter
\def\@seccntformat#1{\csname #1ignore\expandafter\endcsname\csname the#1\endcsname\quad}
\let\sectionignore\@gobbletwo
\let\latex@numberline\numberline
\def\numberline#1{\if\relax#1\relax\else\latex@numberline{#1}\fi}
\makeatother

\makeatletter
\renewcommand\tableofcontents{%
    \@starttoc{toc}%
}
\makeatother

\newcommand{\respacing}{\doublespacing \singlespacing}
\newcommand{\code}[2][1]{\noindent \texttt{\tab[#1] #2} \\}
\newcommand{\skipline}[0]{\texttt{}\\}
\newcommand{\tnl}[0]{\tabularnewline}

\begin{document}
%--------
	\font\titlefont={Times New Roman} at 20pt
	\title{{\titlefont Volt32 CPU}}

	\font\bottomtextfont={Times New Roman} at 12pt
	\date{{\bottomtextfont} \today}
	\author{{\bottomtextfont Andrew Clark}}

	\setmainfont{Times New Roman}
	\setmonofont{Courier New}

	\maketitle
	\pagenumbering{gobble}

	\newpage
	\pagenumbering{arabic}
	%\tableofcontents
	%\newpage

	%\doublespacing

%\section{Abstract}
	%\setcounter{section}{-1}

\section{Table of Contents}
	\tableofcontents
	\newpage

\section{Registers, Main Widths, etc.}
	\begin{itemize}
	%--------
	\item The main width of the processor is 32-bit, and addresses are
	32-bit.  Some 64-bit operations exist.

	\item The machine is an implementation of Line Associative Registers
	(LARs).  Both instruction LARs (ILARs) and data LARs (DLARs) are
	included in the design.  There are a grand total of 128 ILARs and 128
	DLARs, but they are split between the LARs owned by the supervisor
	mode and the LARs owned by the user mode.  There are 64 supervisor mode
	ILARs, 64 supervisor mode DLARs, 64 user mode ILARs, and 64 user mode
	DLARs.

	\item The machine boots in supervisor mode.  The processor jumps to
	address \texttt{0x0} when it enters supervisor mode, which includes
	when the machine boots.

	\item ILARs
		\begin{itemize}
		%--------
		\item In user mode, ILARs 0 to 63 are referred to as \texttt{i0},
		\texttt{i1}, \texttt{i2}, ..., \texttt{i61}, \texttt{i62},
		\texttt{ipc}.

		\item In supervisor mode, ILARs 64 to 127 are referred to as
		\texttt{i0}, \texttt{i1}, \texttt{i2}, ..., \texttt{i61},
		\texttt{i62}, \texttt{ipc}.

		\item The two ILARs called "\texttt{i0}" have all their
		fields set to zero, and when written to, the contents of the ILAR
		does not change.

		\item The two ILARs called "\texttt{ipc}" are the program
		counters for the two operating modes of the processor.

		\item An ILAR's data field is 128 bytes long.  It is composed of
		32-bit instructions aligned to 32 bits.

		\item An ILAR's scalar offset field is 5-bit due to instructions
		being 32-bit and the data field being 128 bytes long.

		\item The base address field of an ILAR is (32 - 7 = 25)-bit.

		\item An ILAR's tag field is 6-bit.
		%--------
		\end{itemize}

	\item DLARs
		\begin{itemize}
		%--------
		\item In user mode, DLARs 0 to 63 are referred to as \texttt{d0},
		\texttt{d1}, \texttt{d2}, ..., \texttt{d61}, \texttt{d62},
		\texttt{d63}.

		\item In supervisor mode, DLARs 64 to 127 are referred to as
		\texttt{d0}, \texttt{d1}, \texttt{d2}, ..., \texttt{d61},
		\texttt{d62}, \texttt{d63}.

		\item The two DLARs called "\texttt{d0}" have all their
		fields set to zero, and when written to, the contents of the DLAR
		does not change.

		\item A DLAR's data field is 128 bytes long.  It is composed of the
		scalar data elements of the 128 byte vectors, where the type of the
		scalar data elements is determined by the type tag field of the
		DLAR.

		\item A DLAR's scalar offset field is 7-bit due to the data field
		being 128 bytes long.

		\item The base address field of a DLAR is (32 - 7 = 25)-bit.

		\item DLARs can take on the following types (3-bit enum):
			\begin{itemize}
			%--------
			\item 8-bit, unsigned (\texttt{u8})
			\item 8-bit, signed (\texttt{i8})
			\item 16-bit, unsigned (\texttt{u16})
			\item 16-bit, signed (\texttt{i16})
			\item 32-bit, unsigned (\texttt{u32})
			\item 32-bit, signed (\texttt{i32})
			\item 64-bit, unsigned (\texttt{u64}); only usable for some
			operations, if not usable gets treated like \texttt{u32}
			\item 64-bit, signed (\texttt{i64}); only usable for some
			operations, if not usable gets treated like \texttt{i32}
			%--------
			\end{itemize}

		\item A DLAR's tag field is 6-bit.
		%--------
		\end{itemize}

	\item The \texttt{ie} register
		\begin{itemize}
		%--------
		\item This register is 1-bit.

		\item This register is a flag indicating whether or not an
		exception can be serviced.  It can be written to with the
		\texttt{cpy} (with \texttt{ie} as the destination operand)
		instruction, and it can be read from with the \texttt{cpy} (with
		\texttt{ie} as a source operand) instruction.

		\item The \texttt{reti} instruction sets \texttt{ie} to
		\texttt{0b1} and returns from supervisor mode to user mode.
		%--------
		\end{itemize}
	%--------
	\end{itemize}

\section{Exceptions}
	Some instructions may cause an exception to occur, putting the
	processor in supervisor mode.

	The following exceptions may occur during normal execution of a
	program.

	\begin{itemize}
	%--------
	\item Taking an interrupt, which also sets \texttt{ie} to \texttt{0b0}.
	\item Division by zero.
	\item \texttt{swi}.
		\begin{itemize}
		%--------
		\item Note that this instruction always causes an exception to
		occur.
		%--------
		\end{itemize}
	\item \texttt{reti} in user mode.
	\item \texttt{cpy} that writes to \texttt{ie} when in user mode.
	\item Instructions that read from/write to supervisor mode
	ILARs or DLARs when in user mode.
	%--------
	\end{itemize}


	%\printbibliography[heading=bibnumbered,title={Bibliography}]

%--------
\end{document}
