\documentclass{article}

\usepackage{pbox}
\usepackage{graphicx}
\usepackage{float}
\usepackage{fancyvrb}
\usepackage[T1]{fontenc}
\usepackage{lmodern}
\usepackage{setspace}
\usepackage[nottoc]{tocbibind}
\usepackage[font=large]{caption}
\usepackage{framed}
\usepackage{tabularx}
\usepackage{amsmath}
\usepackage{hyperref}
\usepackage{fontspec}
\usepackage[backend=biber,sorting=none]{biblatex}
%%\usepackage[
%%	backend=biber,
%%	style=ieee,
%%	sorting=none
%%]{biblatex}
%\addbibresource{project_refs.bib}

%% Hide section, subsection, and subsubsection numbering
%\renewcommand{\thesection}{}
%\renewcommand{\thesubsection}{}
%\renewcommand{\thesubsubsection}{}

% Alternative form of doing section stuff
\renewcommand{\thesection}{}
\renewcommand{\thesubsection}{\arabic{section}.\arabic{subsection}}
\makeatletter
\def\@seccntformat#1{\csname #1ignore\expandafter\endcsname\csname the#1\endcsname\quad}
\let\sectionignore\@gobbletwo
\let\latex@numberline\numberline
\def\numberline#1{\if\relax#1\relax\else\latex@numberline{#1}\fi}
\makeatother

\makeatletter
\renewcommand\tableofcontents{%
    \@starttoc{toc}%
}
\makeatother

\newcommand{\respacing}{\doublespacing \singlespacing}
\newcommand{\code}[2][1]{\noindent \texttt{\tab[#1] #2} \\}
\newcommand{\skipline}[0]{\texttt{}\\}
\newcommand{\tnl}[0]{\tabularnewline}

\begin{document}
%--------
	\font\titlefont={Times New Roman} at 20pt
	\title{{\titlefont Volt32 CPU}}

	\font\bottomtextfont={Times New Roman} at 12pt
	\date{{\bottomtextfont} \today}
	\author{{\bottomtextfont Andrew Clark}}

	\setmainfont{Times New Roman}
	\setmonofont{Courier New}

	\maketitle
	\pagenumbering{gobble}

	\newpage
	\pagenumbering{arabic}
	%\tableofcontents
	%\newpage

	%\doublespacing

%\section{Abstract}
	%\setcounter{section}{-1}

\section{Table of Contents}
	\tableofcontents
	\newpage

\section{Registers, Main Widths, etc.}
	\begin{itemize}
	%--------
	\item The main width of the processor is 32-bit, and addresses are
		32-bit. Some 64-bit operations exist.
	\item The machine is an implementation of Line Associative Registers
		(LARs). Both instruction LARs (ILARs) and data LARs (DLARs) are
		included in the design. There are a grand total of 128 ILARs and
		128 DLARs, but they are split between the LARs owned by the
		supervisor mode and the LARs owned by the user mode. There are 64
		supervisor mode ILARs, 64 supervisor mode DLARs, 64 user mode
		ILARs, and 64 user mode DLARs.
	\item The machine boots in supervisor mode. The processor jumps to
		address \texttt{0x0} when it enters supervisor mode, which includes
		when the machine boots.
	\item ILARs
		\begin{itemize}
		%--------
		\item In user mode, ILARs 0 to 63 are referred to as \texttt{i0},
			\texttt{i1}, \texttt{i2}, ..., \texttt{i61}, \texttt{i62},
			\texttt{ipc}.
		\item In supervisor mode, ILARs 64 to 127 are referred to as
			\texttt{i0}, \texttt{i1}, \texttt{i2}, ..., \texttt{i61},
			\texttt{i62}, \texttt{ipc}.
		\item The two ILARs called "\texttt{i0}" have all their
			fields set to zero, and when written to, the contents of the
			ILAR does not change.
		\item The two ILARs called "\texttt{ipc}" are the program counters
			for the two operating modes of the processor. The location
			within the \texttt{ipc} ILARs can be computed by taking the low
			6 bits (6 because 128 bytes long ILARs) of their addresses.
		\item An ILAR's data field is 128 bytes long. It is composed of
			32-bit instructions aligned to 32 bits.
		\item An ILAR's scalar offset field is (7 - 2 = 5)-bit due to
			instructions being 32-bit and the data field being 128 bytes
			long.
		\item The base address field of an ILAR is (32 - 6 = 26)-bit.
		\item An ILAR's tag field is 7-bit because there are 128 total
			ILARs.
		%--------
		\end{itemize}
	\item DLARs
		\begin{itemize}
		%--------
		\item In user mode, DLARs 0 to 63 are referred to as \texttt{d0},
			\texttt{d1}, \texttt{d2}, ..., \texttt{d59}, \texttt{d60},
			\texttt{dcp}, \texttt{dfp}, \texttt{dsp}.
		\item In supervisor mode, DLARs 64 to 127 are referred to as
			\texttt{d0}, \texttt{d1}, \texttt{d2}, ..., \texttt{d59},
			\texttt{d60}, \texttt{dcp}, \texttt{dfp}, \texttt{dsp}.
		\item The two DLARs called "\texttt{d0}" have all their
			fields set to zero, and when written to, the contents of the
			DLAR does not change.
		\item A DLAR's data field is 128 bytes long. It is composed of the
			scalar data elements of the 128 byte vectors, where the type of
			the scalar data elements is determined by the type tag field of
			the DLAR.
		\item A DLAR's scalar offset field is 7-bit due to the data field
			being 128 bytes long.
		\item The base address field of a DLAR is (32 - 6 = 26)-bit.
		\item DLARs can take on the following types (3-bit enum):
			\begin{itemize}
			%--------
			\item 8-bit, unsigned (\texttt{u8})
			\item 8-bit, signed (\texttt{s8})
			\item 16-bit, unsigned (\texttt{u16})
			\item 16-bit, signed (\texttt{s16})
			\item 32-bit, unsigned (\texttt{u32})
			\item 32-bit, signed (\texttt{s32})
			\item 64-bit, unsigned (\texttt{u64});
				only usable for some operations
			\item 64-bit, signed (\texttt{s64});
				only usable for some operations
			%--------
			\end{itemize}
		\item A DLAR's tag field is 7-bit because there are 128 total
			DLARs.
		\item Similarly, a DLAR's reference count field is 7-bit because
			there are 128 total DLARs.
		\item A DLAR's dirty field is 1-bit.
		%--------
		\end{itemize}
	\item The \texttt{ie} register
		\begin{itemize}
		%--------
		\item "\texttt{ie}" is short for "interrupt enable".
		\item This register is 1-bit.
		\item This register is a flag indicating whether or not an
			interrupt can be serviced. It can be read from/written to using
			\texttt{cpy} instructions.
		\item The \texttt{reti} instruction sets \texttt{ie} to
			\texttt{0b1} and returns to user mode from supervisor mode.
		%--------
		\end{itemize}
	\item The \texttt{xct} register
		\begin{itemize}
		%--------
		\item "exception type"
		\item This register is 32-bit.
		\item This register is set to the numerical value of an exception's
			type upon the machine entering supervisor mode. It can be read
			from/written to using \texttt{cpy} instructions.
		%--------
		\end{itemize}
	\item The \texttt{swiarg0} register
		\begin{itemize}
		%--------
		\item This register is 32-bit.
		\item This register indicates argument 0 to \texttt{swi}. It can
			be read from/written to with \texttt{cpy} instructions.
		%--------
		\end{itemize}
	\item The \texttt{swiarg1} register
		\begin{itemize}
		%--------
		\item This register is 32-bit.
		\item This register indicates argument 1 to \texttt{swi}. It can
			be read from/written to with \texttt{cpy} instructions.
		%--------
		\end{itemize}
	\item The \texttt{swiarg2} register
		\begin{itemize}
		%--------
		\item This register is 32-bit.
		\item This register indicates argument 2 to \texttt{swi}. It can
			be read from/written to with \texttt{cpy} instructions.
		%--------
		\end{itemize}
	\item The \texttt{swiarg3} register
		\begin{itemize}
		%--------
		\item This register is 32-bit.
		\item This register indicates argument 3 to \texttt{swi}. It can
			be read from/written to with \texttt{cpy} instructions.
		%--------
		\end{itemize}
	%--------
	\end{itemize}
	\newpage
\section{Exceptions}
	Some instructions may cause an exception to occur, putting the
	processor in supervisor mode.

	The following exceptions may occur during normal execution of a
	program. \texttt{xct} is set to a numerical value representing these
	upon the processor encountering an exception. 

	\begin{itemize}
	%--------
	\item When \texttt{xct == 0x0}:
		Taking a non-software interrupt (which would also set \texttt{ie}
		to \texttt{0b0}).
	\item When \texttt{xct == 0x1}: Division by zero.
	\item When \texttt{xct == 0x2}: Undefined instruction.
	\item When \texttt{xct == 0x3}:
		Instructions where 64-bit ops are not defined.
	\item When \texttt{xct == 0x4}: \texttt{swi}.
	\item When \texttt{xct == 0x5}: \texttt{reti} when in user mode.
	\item When \texttt{xct == 0x6}: \texttt{retx} when in user mode.

	\item When \texttt{xct == 0x7}:
		\texttt{cpy} that reads from \texttt{ie} in user mode.
	\item When \texttt{xct == 0x8}:
		\texttt{cpy} that writes to \texttt{ie} in user mode.
	\item When \texttt{xct == 0x9}:
		\texttt{cpy} that reads from \texttt{xct} in user mode.
	\item When \texttt{xct == 0xa}:
		\texttt{cpy} that writes to \texttt{xct} in user mode.

	% I need to not bring these `swiargN` `cpy`s back to causing
	% exceptions, or at least not the ones that *write* to these registers.
	% Not being able to write 

	%\item \texttt{cpy} that reads from \texttt{swiarg0} when in user mode.
	%\item \texttt{cpy} that writes to \texttt{swiarg0} when in user mode.
	%\item \texttt{cpy} that reads from \texttt{swiarg1} when in user mode.
	%\item \texttt{cpy} that writes to \texttt{swiarg1} when in user mode.

	%\item \texttt{cpy} that reads from \texttt{swiarg2} when in user mode.
	%\item \texttt{cpy} that writes to \texttt{swiarg2} when in user mode.
	%\item \texttt{cpy} that reads from \texttt{swiarg3} when in user mode.
	%\item \texttt{cpy} that writes to \texttt{swiarg3} when in user mode.

	\item When \texttt{xct == 0xb}:
		Instructions that read from supervisor mode ILARs or DLARs when in
		user mode.
	\item When \texttt{xct == 0xc}:
		Instructions that write to supervisor mode ILARs or DLARs when in
		user mode.
	%--------
	\end{itemize}
	\newpage
\section{Instructions}
	\subsection{Group 0 Instructions}
		\begin{itemize}
		%--------
		\item Encoding: \texttt{0000 aaaa aabb bbbb  cccc ccdd dddv oooo}
			\begin{itemize}
			%--------
			\item \texttt{a}: DLAR a
			\item \texttt{b}: DLAR b
			\item \texttt{c}: DLAR c
			\item \texttt{d}: DLAR d (\texttt{d0} to \texttt{d31})
			\item \texttt{v}:
				\begin{itemize}
				%--------
				\item when \texttt{0b0}: scalar operation. The assembly
					syntax indicating a scalar operation simply adds
					"\texttt{.s}" to the instruction's name.
				\item when \texttt{0b1}: vector operation. The assembly
					syntax indicating a vector operation simply adds
					"\texttt{.v}" to the instruction's name.
				%--------
				\end{itemize}
			\item \texttt{o}: Opcode
			%--------
			\end{itemize}
		\item Instruction List:
			\begin{enumerate}
			%--------
			\item \texttt{add dA, dB, dC}
				%\begin{itemize}
				%%--------
				%\item This instruction causes an exception if \texttt{dA}
				%	is of the following types: \texttt{u64}, \texttt{s64}.
				%%--------
				%\end{itemize}
			\item \texttt{sub dA, dB, dC}
				%\begin{itemize}
				%%--------
				%\item This instruction causes an exception if \texttt{dA}
				%	is of the following types: \texttt{u64}, \texttt{s64}.
				%%--------
				%\end{itemize}
			\item \texttt{slt dA, dB, dC}
				%\begin{itemize}
				%%--------
				%\item This instruction causes an exception if \texttt{dA}
				%	is of the following types: \texttt{u64}, \texttt{s64}.
				%%--------
				%\end{itemize}
			\item \texttt{mul dA, dB, dC}
				\begin{itemize}
				%--------
				\item This instruction causes an exception if \texttt{dB}
					or \texttt{dC} is of the following types: \texttt{u64},
					\texttt{s64}.
				%--------
				\end{itemize}

			\item \texttt{div dA, dB, dC, dD}
				\begin{itemize}
				%--------
				\item This instruction causes an exception if \texttt{dB},
					\texttt{dC}, or \texttt{dD} is of the following types:
					\texttt{u64}, \texttt{s64}.

				\item This instruction writes the quotient into
					\texttt{dA}, and the remainder into \texttt{dD}.
				%--------
				\end{itemize}
			\item \texttt{and dA, dB, dC}
			\item \texttt{or dA, dB, dC}
			\item \texttt{xor dA, dB, dC}

			\item \texttt{shl dA, dB, dC}
				\begin{itemize}
				%--------
				\item Logical shift left.
				\item This instruction causes an exception if \texttt{dA},
					\texttt{dB}, or \texttt{dC} is of the following types:
					\texttt{u64}, \texttt{s64}.
				\item This instruction casts a temporary copy of
					\texttt{dC} to the unsigned type that is the same size
					as \texttt{dA}'s type and uses that instead of
					\texttt{dC}.
				%--------
				\end{itemize}
			\item \texttt{shr dA, dB, dC}
				\begin{itemize}
				%--------
				\item Logical shift right if \texttt{dA} is unsigned, or
					arithmetic shift right if \texttt{dA} is signed.
				\item This instruction causes an exception if \texttt{dA},
					\texttt{dB}, or \texttt{dC} is of the following types:
					\texttt{u64}, \texttt{s64}.
				\item This instruction casts a temporary copy of
					\texttt{dC} to the unsigned type that is the same size
					as \texttt{dA}'s type and uses that instead of
					\texttt{dC}.
				%--------
				\end{itemize}
			\item \texttt{rol dA, dB, dC}
				\begin{itemize}
				%--------
				\item Rotate left.
				\item This instruction causes an exception if \texttt{dA},
					\texttt{dB}, or \texttt{dC} is of the following types:
					\texttt{u64}, \texttt{s64}.
				\item This instruction casts a temporary copy of
					\texttt{dC} to the unsigned type that is the same size
					as \texttt{dA}'s type and uses that instead of
					\texttt{dC}.
				%--------
				\end{itemize}
			\item \texttt{ror dA, dB, dC}
				\begin{itemize}
				%--------
				\item Rotate right.
				\item This instruction causes an exception if \texttt{dA},
					\texttt{dB}, or \texttt{dC} is of the following types:
					\texttt{u64}, \texttt{s64}.
				\item This instruction casts a temporary copy of
					\texttt{dC} to the unsigned type that is the same size
					as \texttt{dA}'s type and uses that instead of
					\texttt{dC}.
				%--------
				\end{itemize}

			\item \texttt{add dA, dB.addr, dC}
				\begin{itemize}
				%--------
				\item This instruction causes an exception if \texttt{dA}
					or \texttt{dC} is of the following types: \texttt{u64},
					\texttt{s64}.
				\item For \texttt{add.v dA, dB.addr, dC}, this instruction
					duplicates the value of \texttt{cast(dA.type, dB.addr)}
					(a scalar) into a temporary (64 bytes long) vector of
					element type \texttt{dA.type}.
				%--------
				\end{itemize}
			\item \texttt{shl dA, dB.addr, dC}
				\begin{itemize}
				%--------
				\item This instruction causes an exception if \texttt{dA}
					or \texttt{dC} is of the following types: \texttt{u64},
					\texttt{s64}.
				\item For \texttt{shl.v dA, dB.addr, dC}, this instruction
					duplicates \texttt{cast(dA.type, dB.addr)} (a scalar)
					into a temporary (64 bytes long) vector of element type
					\texttt{dA.type}.
				\item This instruction casts a temporary copy of
					\texttt{dC} to the unsigned type that is the same size
					as \texttt{dA}'s type and uses that instead of
					\texttt{dC}.
				%--------
				\end{itemize}
			\item \textit{Reserved for future expansion.}
			\item \textit{Reserved for future expansion.}
			%--------
			\end{enumerate}
		%--------
		\end{itemize}
		\newpage
	\subsection{Group 1 Instructions}
		\begin{itemize}
		%--------
		\item Encoding: \texttt{0001 aaaa aabb bbbb  0000 0000 0000 oooo}
			\begin{itemize}
			%--------
			\item \texttt{a}: DLAR a
			\item \texttt{b}: DLAR b
			\item \texttt{o}: Opcode
			%--------
			\end{itemize}

		\item Instruction List:
			\begin{enumerate}
			%--------
			\item \texttt{add.r dA, dB}
				\begin{itemize}
				%--------
				\item This instruction casts a temporary copy of
					\texttt{dB} to the \texttt{dA}'s type and performs a
					sum of all the scalar data elements of the temporary
					copy of \texttt{dB}, then stores the result in
					\texttt{dA}'s scalar data.
				\item This instruction causes an exception if \texttt{dA}
					is of the following types: \texttt{u64}, \texttt{s64}.
				%--------
				\end{itemize}
			\item \texttt{mul.r dA, dB}
				\begin{itemize}
				%--------
				\item This instruction casts a temporary copy of
					\texttt{dB} to the \texttt{dA}'s type and performs a
					product of all the scalar data elements of the
					temporary copy of \texttt{dB}, then stores the result
					in \texttt{dA}'s scalar data.
				\item This instruction causes an exception if \texttt{dA}
					is of the following types: \texttt{u64}, \texttt{s64}.
				%--------
				\end{itemize}
			\item \texttt{max.r dA, dB}
				\begin{itemize}
				%--------
				\item This instruction casts a temporary copy of
					\texttt{dB} to the \texttt{dA}'s type and finds the
					scalar data element of the temporary copy of
					\texttt{dB} that is the maximum, then stores the result
					in \texttt{dA}'s scalar data.
				\item This instruction causes an exception if \texttt{dA}
					is of the following types: \texttt{u64}, \texttt{s64}.
				%--------
				\end{itemize}
			\item \texttt{min.r dA, dB}
				\begin{itemize}
				%--------
				\item This instruction casts a temporary copy of
					\texttt{dB} to the \texttt{dA}'s type and finds the
					scalar data element of the temporary copy of
					\texttt{dB} that is the minimum, then stores the result
					in \texttt{dA}'s scalar data.
				\item This instruction causes an exception if \texttt{dA}
					is of the following types: \texttt{u64}, \texttt{s64}.
				%--------
				\end{itemize}

			\item \texttt{and.r dA, dB}
				\begin{itemize}
				%--------
				\item This instruction casts a temporary copy of
					\texttt{dB} to the \texttt{dA}'s type and performs a
					bitwise AND reduction of all the scalar data elements
					of the temporary copy of \texttt{dB}, then stores the
					result in \texttt{dA}'s scalar data.
				\item This instruction causes an exception if \texttt{dA}
					is of the following types: \texttt{u64}, \texttt{s64}.
				%--------
				\end{itemize}
			\item \texttt{or.r dA, dB}
				\begin{itemize}
				%--------
				\item This instruction casts a temporary copy of
					\texttt{dB} to the \texttt{dA}'s type and performs a
					bitwise OR reduction of all the scalar data elements of
					the temporary copy of \texttt{dB}, then stores the
					result in \texttt{dA}'s scalar data.
				\item This instruction causes an exception if \texttt{dA}
					is of the following types: \texttt{u64}, \texttt{s64}.
				%--------
				\end{itemize}
			\item \texttt{xor.r dA, dB}
				\begin{itemize}
				%--------
				\item This instruction casts a temporary copy of
					\texttt{dB} to the \texttt{dA}'s type and performs a
					bitwise XOR reduction of all the scalar data elements
					of the temporary copy of \texttt{dB}, then stores the
					result in \texttt{dA}'s scalar data.
				\item This instruction causes an exception if \texttt{dA}
					is of the following types: \texttt{u64}, \texttt{s64}.
				%--------
				\end{itemize}
			\item \textit{Reserved for future expansion.}

			\item \textit{Reserved for future expansion.}
			\item \textit{Reserved for future expansion.}
			\item \textit{Reserved for future expansion.}
			\item \textit{Reserved for future expansion.}

			\item \textit{Reserved for future expansion.}
			\item \textit{Reserved for future expansion.}
			\item \textit{Reserved for future expansion.}
			\item \textit{Reserved for future expansion.}
			%--------
			\end{enumerate}
		%--------
		\end{itemize}
		\newpage

	\subsection{Group 2 Instructions}
		\begin{itemize}
		%--------
		\item Encoding: \texttt{0010 aaaa aabb bbbb  cccc ccii iiii iooo}
			\begin{itemize}
			%--------
			\item \texttt{a}: DLAR a
			\item \texttt{b}: DLAR b
			\item \texttt{c}: DLAR c
			\item \texttt{i}: \texttt{simm7} (sign-extended 7-bit
				immediate)
			\item \texttt{o}: Opcode
			%--------
			\end{itemize}
		\item For these instructions, the \texttt{dB} register's scalar
			data field (temporarily casted to the \texttt{u32} type) and
			the \texttt{dC} register's address field are used. 
		\item Also, the type that a DLAR is set to is indicated in the
			instruction name, with, for example, \texttt{ldu8} setting
			\texttt{dA}'s type to \texttt{u8}.
		\item Instruction List:
			\begin{enumerate}
			%--------
			\item \texttt{ldu8 dA, dB, dC, simm7}
			\item \texttt{lds8 dA, dB, dC, simm7}
			\item \texttt{ldu16 dA, dB, dC, simm7}
			\item \texttt{lds16 dA, dB, dC, simm7}
			\item \texttt{ldu32 dA, dB, dC, simm7}
			\item \texttt{lds32 dA, dB, dC, simm7}
			\item \texttt{ldu64 dA, dB, dC, simm7}
			\item \texttt{lds64 dA, dB, dC, simm7}
			%--------
			\end{enumerate}
		%--------
		\end{itemize}
		\newpage

	\subsection{Group 3 Instructions}
		\begin{itemize}
		%--------
		\item Encoding: \texttt{0011 aaaa aabb bbbb  cccc ccii iiii iooo}
			\begin{itemize}
			%--------
			\item \texttt{a}: DLAR a
			\item \texttt{b}: DLAR b
			\item \texttt{c}: DLAR c
			\item \texttt{i}: \texttt{simm7} (sign-extended 7-bit
				immediate)
			\item \texttt{o}: Opcode
			%--------
			\end{itemize}
		\item For these instructions, the \texttt{dB} register's scalar
			data field (temporarily casted to the \texttt{u32} type) and
			the \texttt{dC} register's address field are used. 
		\item Also, the type that a DLAR is set to is indicated in the
			instruction name, with, for example, \texttt{stu8} setting
			\texttt{dA}'s type to \texttt{u8}.
		\item Instruction List:
			\begin{enumerate}
			%--------
			\item \texttt{stu8 dA, dB, dC, simm7}
			\item \texttt{sts8 dA, dB, dC, simm7}
			\item \texttt{stu16 dA, dB, dC, simm7}
			\item \texttt{sts16 dA, dB, dC, simm7}
			\item \texttt{stu32 dA, dB, dC, simm7}
			\item \texttt{sts32 dA, dB, dC, simm7}
			\item \texttt{stu64 dA, dB, dC, simm7}
			\item \texttt{sts64 dA, dB, dC, simm7}
			%--------
			\end{enumerate}
		%--------
		\end{itemize}
		\newpage

	\subsection{Group 4 Instructions}
		\begin{itemize}
		%--------
		\item Encoding: \texttt{0100 aaaa aabb bbbb  0000 0000 0000 0ooo}
			\begin{itemize}
			%--------
			\item \texttt{a}: DLAR a
			\item \texttt{b}: DLAR b
			\item \texttt{o}: Opcode
			%--------
			\end{itemize}
		\item For these instructions, the \texttt{dB} register's scalar
			data field is used.
		\item Instruction List:
			\begin{enumerate}
			%--------
			\item \texttt{dpu8 dA, dB}
				\begin{itemize}
				%--------
				\item This instruction casts (a temporary copy of) the
					scalar data of \texttt{dB} to the \texttt{u8} type. The
					casted scalar data is then stored into every
					\texttt{u8} vector element of \texttt{dA}. The type of
					\texttt{dA} is then changed to \texttt{u8}.
				%--------
				\end{itemize}
			\item \texttt{dps8 dA, dB}
				\begin{itemize}
				%--------
				\item This instruction casts (a temporary copy of) the
					scalar data of \texttt{dB} to the \texttt{s8} type. The
					casted scalar data is then stored into every
					\texttt{s8} vector element of \texttt{dA}. The type of
					\texttt{dA} is then changed to \texttt{s8}.
				%--------
				\end{itemize}
			\item \texttt{dpu16 dA, dB}
				\begin{itemize}
				%--------
				\item This instruction casts (a temporary copy of) the
					scalar data of \texttt{dB} to the \texttt{u16} type.
					The casted scalar data is then stored into every
					\texttt{u16} vector element of \texttt{dA}. The type of
					\texttt{dA} is then changed to \texttt{u16}.
				%--------
				\end{itemize}
			\item \texttt{dps16 dA, dB}
				\begin{itemize}
				%--------
				\item This instruction casts (a temporary copy of) the
					scalar data of \texttt{dB} to the \texttt{s16} type.
					The casted scalar data is then stored into every
					\texttt{s16} vector element of \texttt{dA}. The type of
					\texttt{dA} is then changed to \texttt{s16}.
				%--------
				\end{itemize}

			\item \texttt{dpu32 dA, dB}
				\begin{itemize}
				%--------
				\item This instruction casts (a temporary copy of) the
					scalar data of \texttt{dB} to the \texttt{u32} type.
					The casted scalar data is then stored into every
					\texttt{u32} vector element of \texttt{dA}. The type of
					\texttt{dA} is then changed to \texttt{u32}.
				%--------
				\end{itemize}
			\item \texttt{dps32 dA, dB}
				\begin{itemize}
				%--------
				\item This instruction casts (a temporary copy of) the
					scalar data of \texttt{dB} to the \texttt{s32} type.
					The casted scalar data is then stored into every
					\texttt{s32} vector element of \texttt{dA}. The type of
					\texttt{dA} is then changed to \texttt{s32}.
				%--------
				\end{itemize}
			\item \texttt{dpu64 dA, dB}
				\begin{itemize}
				%--------
				\item This instruction casts (a temporary copy of) the
					scalar data of \texttt{dB} to the \texttt{u64} type.
					The casted scalar data is then stored into every
					\texttt{u64} vector element of \texttt{dA}. The type of
					\texttt{dA} is then changed to \texttt{u64}.
				%--------
				\end{itemize}
			\item \texttt{dps64 dA, dB}
				\begin{itemize}
				%--------
				\item This instruction casts (a temporary copy of) the
					scalar data of \texttt{dB} to the \texttt{s64} type.
					The casted scalar data is then stored into every
					\texttt{s64} vector element of \texttt{dA}. The type of
					\texttt{dA} is then changed to \texttt{s64}.
				%--------
				\end{itemize}
			%--------
			\end{enumerate}
		%--------
		\end{itemize}
		\newpage

	\subsection{Group 5 Instructions}
		\begin{itemize}
		%--------
		\item Encoding: \texttt{0101 aaaa aabb bbbb  cccc ccii iiii jjjj}
			\begin{itemize}
			%--------
			\item \texttt{a}: ILAR a
			\item \texttt{b}: ILAR b
			\item \texttt{c}: DLAR c
			\item \texttt{i}: \texttt{isimm6} (sign-extended 6-bit
				immediate)
			\item \texttt{j}: \texttt{jimm4} (zero-extended 4-bit
				immediate), the number of consecutive ILARs past
				\texttt{iA} to \texttt{fetch} into.
			%--------
			\end{itemize}

		\item This instruction uses the address field of \texttt{iB} and
			the scalar data field (temporarily casted to the \texttt{u32}
			type) of the \texttt{dC} DLAR.
		\item Also, the address to \texttt{fetch} from is computed by
			adding these two temporary values to a temporary value
			equivalent to the sign extended value of \texttt{isimm6}.

		\item Instruction List:
			\begin{enumerate}
			%--------
			\item \texttt{fetch iA, iB, dC, isimm6, jimm4}
			%--------
			\end{enumerate}
		%--------
		\end{itemize}
		\newpage

	\subsection{Group 6 Instructions}
		\begin{itemize}
		%--------
		%\item Encoding: \texttt{0110 aaaa aabb bbbb  cccc ccdd dddv oooo}
		\item Encoding: \texttt{0110 aaaa aabb bbbb  0iii iijj jjjv oooo}
			\begin{itemize}
			%--------
			\item \texttt{a}: DLAR a
			\item \texttt{b}: ILAR b
			%\item \texttt{c}: DLAR c
			\item \texttt{i}: \texttt{iimm5} (zero-extended 5-bit
				immediate \texttt{i})
			\item \texttt{j}: \texttt{jimm5} (zero-extended 5-bit
				immediate \texttt{j})
			\item \texttt{v}:
				\begin{itemize}
				%--------
				\item when \texttt{0b0}: scalar operation (uses the scalar
					data of \texttt{dA}). The assembly syntax indicating a
					scalar operation simply adds "\texttt{.s}" to the
					instruction's name.
				\item when \texttt{0b1}: vector operation (uses the vector
					data of \texttt{dA}). The assembly syntax indicating a
					vector operation simply adds "\texttt{.v}" to the
					instruction's name.
				%--------
				\end{itemize}
			\item \texttt{o}: Opcode
			%--------
			\end{itemize}
		\item These instructions use the scalar or vector data field of
			\texttt{dA}.
		\item Instruction List:
			\begin{enumerate}
			%--------
			\item \texttt{sel dA, iB, iimm5, jimm5}
				\begin{itemize}
				%--------
				\item This instruction jumps to \texttt{iB[iimm5 << 2]} if
					the particular data field of \texttt{dA} is non-zero,
					otherwise to the address \texttt{iB[jimm5 << 2]}.
				%--------
				\end{itemize}
			\item \texttt{jz dA, iB, iimm5}
				\begin{itemize}
				%--------
				\item This instruction jumps to \texttt{iB[iimm5 << 2]}
					if the particular data field of \texttt{dA} is zero.
				%--------
				\end{itemize}
			\item \texttt{jnz dA, iB, iimm5}
				\begin{itemize}
				%--------
				\item This instruction jumps to \texttt{iB[iimm5 << 2]}
					if the particular data field of \texttt{dA} is
					non-zero.
				%--------
				\end{itemize}
			\item \texttt{reti dA}
				\begin{itemize}
				%--------
				\item This instruction returns from an interrupt if
					\texttt{dA} is non-zero, setting \texttt{ie} to
					\texttt{0b1}.
				\item This instruction causes an exception if the processor
					is in user mode.
				%--------
				\end{itemize}

			\item \texttt{retx dA}
				\begin{itemize}
				%--------
				\item This instruction returns from supervisor mode to user
					mode if \texttt{dA} is non-zero.
				\item This instruction causes an exception if the processor
					is in user mode.
				%--------
				\end{itemize}
			\item \textit{Reserved for future expansion.}
			\item \textit{Reserved for future expansion.}
			\item \textit{Reserved for future expansion.}

			\item \textit{Reserved for future expansion.}
			\item \textit{Reserved for future expansion.}
			\item \textit{Reserved for future expansion.}
			\item \textit{Reserved for future expansion.}

			\item \textit{Reserved for future expansion.}
			\item \textit{Reserved for future expansion.}
			\item \textit{Reserved for future expansion.}
			\item \textit{Reserved for future expansion.}
			%--------
			\end{enumerate}
		%--------
		\end{itemize}
		\newpage

	\subsection{Group 7 Instructions}
		\begin{itemize}
		%--------
		\item Encoding: \texttt{0111 aaaa aabb bbbb  cccc ccii iiii sooo}
			\begin{itemize}
			%--------
			\item \texttt{a}: ILAR a or DLAR a
			\item \texttt{b}: ILAR b or DLAR b
			\item \texttt{c}: DLAR c
			\item \texttt{i}: \texttt{imm6} (zero-extended 6-bit
				immediate), amount of LAR contents to use
			\item \texttt{s}:
				\begin{itemize}
				%--------
				\item when \texttt{0b0}: Destination LARs (the ones
					starting with \texttt{iA} or \texttt{dA}) or source
					LARs (the ones starting with \texttt{iB} or
					\texttt{dB}) are user mode LARs. An example of the
					syntax for \texttt{getaddr} in this case is
					\texttt{getaddr.u}. The "\texttt{.u}" suffix indicates
					this instruction will have the \texttt{s} bit set to
					\texttt{0b0}.
				\item when \texttt{0b1}: Destination LARs (the ones
					starting with \texttt{iA} or \texttt{dA}) or source
					LARs (the ones starting with \texttt{iB} or
					\texttt{dB}) are supervisor mode LARs. An example of
					the syntax for \texttt{getaddr} in this case is
					\texttt{getaddr.s}. The "\texttt{.s}" suffix indicates
					this instruction will have the \texttt{s} bit set to
					\texttt{0b1}.
				%--------
				\end{itemize}
			\item \texttt{o}: Opcode
			%--------
			\end{itemize}
		\item Instruction List:
			\begin{enumerate}
			%--------
			\item \texttt{getaddr dA, dB, imm6}
				\begin{itemize}
				%--------
				\item This instruction uses the \texttt{s} bit to indicate
					which mode the source DLARs belong to.
				\item This instruction grabs the addresses of source DLARs
					starting with \texttt{dB} and then also the following
					\texttt{imm6 - 1} source DLARs. The grabbed addresses
					are then written into consecutive scalar data elements
					of destination DLARs (starting with \texttt{dA} and
					continuing into the following destination DLARs as
					necessary).
				\item When supervisor mode LARs are used for the source(s),
					this instruction causes an exception if used in user
					mode.
				%--------
				\end{itemize}
			\item \texttt{getaddr dA, iB, imm6}
				\begin{itemize}
				%--------
				\item This instruction uses the \texttt{s} bit to indicate
					which mode the source ILARs belong to.
				\item This instruction grabs the addresses of source ILARs
					starting with \texttt{dB} and then also the following
					\texttt{imm6 - 1} source DLARs. The grabbed addresses
					are then written into consecutive scalar data elements
					of destination DLARs (starting with \texttt{dA} and
					continuing into the following destination DLARs as
					necessary).
				\item When supervisor mode LARs are used for the source(s),
					this instruction causes an exception if used in user
					mode.
				%--------
				\end{itemize}
			\item \texttt{gettype dA, dB, imm6}
				\begin{itemize}
				%--------
				\item This instruction uses the \texttt{s} bit to indicate
					which mode the source DLARs belong to.
				\item This instruction grabs the types of source DLARs
					starting with \texttt{dB} and then also the following
					\texttt{imm6 - 1} source DLARs. The grabbed types are
					then written into consecutive scalar data elements of
					destination DLARs (starting with \texttt{dA} and
					continuing into the following destination DLARs as
					necessary).
				\item When supervisor mode LARs are used for the source(s),
					this instruction causes an exception if used in user
					mode.
				%--------
				\end{itemize}
			\item \texttt{ldm dA, dB, dC, imm6}
				\begin{itemize}
				%--------
				\item This instruction's name is short for "load multiple".
				\item This instruction uses the \texttt{s} bit to indicate
					to which mode the destination DLARs belong.
				\item This instruction uses addresses stored in the
					\texttt{imm6} scalar data elements of consecutive DLARs
					(starting with \texttt{dB}) and types stored in the
					\texttt{imm6} scalar data elements of consecutive DLARs
					(starting with \texttt{dC}). Multiple loads from memory
					are performed into the \texttt{imm6} destination DLARs
					(starting with \texttt{dA}).
				\item When supervisor mode LARs are used for the
					destination(s), this instruction causes an exception if
					used in user mode.
				%--------
				\end{itemize}

			\item \texttt{fetchm iA, dB, imm6}
				\begin{itemize}
				%--------
				\item This instruction's name is short for "fetch
					multiple".
				\item This instruction uses the \texttt{s} bit to indicate
					to which mode the destination DLARs belong.
				\item This instruction uses addresses stored in the
					\texttt{imm6} scalar data elements of consecutive DLARs
					(starting with \texttt{dB}). Multiple fetches from
					memory are performed into the \texttt{imm6} destination
					ILARs (starting with \texttt{iA}).
				\item When supervisor mode LARs are used for the
					destination(s), this instruction causes an exception if
					used in user mode.
				%--------
				\end{itemize}
			\item \textit{Reserved for future expansion.}
			\item \textit{Reserved for future expansion.}
			\item \textit{Reserved for future expansion.}
			%--------
			\end{enumerate}
		%--------
		\end{itemize}
		\newpage

	\subsection{Group 8 Instructions}
		\begin{itemize}
		%--------
		\item Encoding: \texttt{1000 aaaa aabb bbbb  cccc ccdd dddd oooo}
			\begin{itemize}
			%--------
			\item \texttt{a}: DLAR a
			\item \texttt{b}: DLAR b
			\item \texttt{c}: DLAR c
			\item \texttt{d}: DLAR d
			\item \texttt{o}: Opcode
			%--------
			\end{itemize}

		\item Instruction List:
			\begin{enumerate}
			%--------
			\item \texttt{cpy dA, ie}
			\item \texttt{cpy ie, dA}
			\item \texttt{cpy dA, xct}
			\item \texttt{cpy xct, dA}

			\item \texttt{cpy dA, swiarg0}
			\item \texttt{cpy swiarg0, dA}
			\item \texttt{cpy dA, swiarg1}
			\item \texttt{cpy swiarg1, dA}

			\item \texttt{cpy dA, swiarg2}
			\item \texttt{cpy swiarg2, dA}
			\item \texttt{cpy dA, swiarg3}
			\item \texttt{cpy swiarg3, dA}

			\item \texttt{swi dA, dB, dC, dD}
				\begin{itemize}
				%--------
				\item Note that this instruction always causes an exception
					to occur.
				%--------
				\end{itemize}
			\item \textit{Reserved for future expansion.}
			\item \textit{Reserved for future expansion.}
			\item \textit{Reserved for future expansion.}
			%--------
			\end{enumerate}
		%--------
		\end{itemize}

	\subsection{Group 9 Instructions}
		\begin{itemize}
		%--------
		\item Encoding: \texttt{1001 aaaa aabb bbbb  cccc cc00 000v oooo}
			\begin{itemize}
			%--------
			\item \texttt{a}: DLAR a
			\item \texttt{b}: DLAR b
			\item \texttt{c}: DLAR c
			\item \texttt{v}:
				\begin{itemize}
				%--------
				\item when \texttt{0b0}: scalar operation. The assembly
					syntax indicating a scalar operation simply adds
					"\texttt{.s}" to the instruction's name.
				\item when \texttt{0b1}: vector operation. The assembly
					syntax indicating a vector operation simply adds
					"\texttt{.v}" to the instruction's name.
				%--------
				\end{itemize}
			\item \texttt{o}: Opcode
			%--------
			\end{itemize}
		\item These instructions perform a read from/write to the IO
			address calculated by adding the scalar data of \texttt{dB} and
			\texttt{dC}, or (in other words) \texttt{dB + dC}. The types of
			\texttt{dB} and \texttt{dC} are temporarily casted to
			\texttt{u32} for this calculation.
		\item When a scalar operation is being performed, only the
			scalar data of \texttt{dA} is read into from/written out to IO
			space.
		\item When a vector operation is being performed, the
			entire vector data of \texttt{dA} is read into from/written out
			to IO space.
		\item Instruction List:
			\begin{enumerate}
			%--------
			\item \texttt{inu8 dA, dB, dC}
				\begin{itemize}
				%--------
				\item This instruction sets the type of \texttt{dA} to
					\texttt{u8} before performing anything else of the
					operation.
				\item This instruction reads from IO space and writes to
					\texttt{dA}.
				%--------
				\end{itemize}
			\item \texttt{ins8 dA, dB, dC}
				\begin{itemize}
				%--------
				\item This instruction sets the type of \texttt{dA} to
					\texttt{s8} before performing anything else of the
					operation.
				\item This instruction reads from IO space and writes to
					\texttt{dA}.
				%--------
				\end{itemize}
			\item \texttt{inu16 dA, dB, dC}
				\begin{itemize}
				%--------
				\item This instruction sets the type of \texttt{dA} to
					\texttt{u16} before performing anything else of the
					operation.
				\item This instruction reads from IO space and writes to
					\texttt{dA}.
				%--------
				\end{itemize}
			\item \texttt{ins16 dA, dB, dC}
				\begin{itemize}
				%--------
				\item This instruction sets the type of \texttt{dA} to
					\texttt{s16} before performing anything else of the
					operation.
				\item This instruction reads from IO space and writes to
					\texttt{dA}.
				%--------
				\end{itemize}

			\item \texttt{inu32 dA, dB, dC}
				\begin{itemize}
				%--------
				\item This instruction sets the type of \texttt{dA} to
					\texttt{u32} before performing anything else of the
					operation.
				\item This instruction reads from IO space and writes to
					\texttt{dA}.
				%--------
				\end{itemize}
			\item \texttt{ins32 dA, dB, dC}
				\begin{itemize}
				%--------
				\item This instruction sets the type of \texttt{dA} to
					\texttt{s32} before performing anything else of the
					operation.
				\item This instruction reads from IO space and writes to
					\texttt{dA}.
				%--------
				\end{itemize}
			\item \texttt{inu64 dA, dB, dC}
				\begin{itemize}
				%--------
				\item This instruction sets the type of \texttt{dA} to
					\texttt{u64} before performing anything else of the
					operation.
				\item This instruction reads from IO space and writes to
					\texttt{dA}.
				%--------
				\end{itemize}
			\item \texttt{ins64 dA, dB, dC}
				\begin{itemize}
				%--------
				\item This instruction sets the type of \texttt{dA} to
					\texttt{s64} before performing anything else of the
					operation.
				\item This instruction reads from IO space and writes to
					\texttt{dA}.
				%--------
				\end{itemize}

			\item \texttt{outu8 dA, dB, dC}
				\begin{itemize}
				%--------
				\item This instruction sets the type of \texttt{dA} to
					\texttt{u8} before performing anything else of the
					operation.
				\item This instruction writes to IO space and reads from
					\texttt{dA}.
				%--------
				\end{itemize}
			\item \texttt{outs8 dA, dB, dC}
				\begin{itemize}
				%--------
				\item This instruction sets the type of \texttt{dA} to
					\texttt{s8} before performing anything else of the
					operation.
				\item This instruction writes to IO space and reads from
					\texttt{dA}.
				%--------
				\end{itemize}
			\item \texttt{outu16 dA, dB, dC}
				\begin{itemize}
				%--------
				\item This instruction sets the type of \texttt{dA} to
					\texttt{u16} before performing anything else of the
					operation.
				\item This instruction writes to IO space and reads from
					\texttt{dA}.
				%--------
				\end{itemize}
			\item \texttt{outs16 dA, dB, dC}
				\begin{itemize}
				%--------
				\item This instruction sets the type of \texttt{dA} to
					\texttt{s16} before performing anything else of the
					operation.
				\item This instruction writes to IO space and reads from
					\texttt{dA}.
				%--------
				\end{itemize}

			\item \texttt{outu32 dA, dB, dC}
				\begin{itemize}
				%--------
				\item This instruction sets the type of \texttt{dA} to
					\texttt{u32} before performing anything else of the
					operation.
				\item This instruction writes to IO space and reads from
					\texttt{dA}.
				%--------
				\end{itemize}
			\item \texttt{outs32 dA, dB, dC}
				\begin{itemize}
				%--------
				\item This instruction sets the type of \texttt{dA} to
					\texttt{s32} before performing anything else of the
					operation.
				\item This instruction writes to IO space and reads from
					\texttt{dA}.
				%--------
				\end{itemize}
			\item \texttt{outu64 dA, dB, dC}
				\begin{itemize}
				%--------
				\item This instruction sets the type of \texttt{dA} to
					\texttt{u64} before performing anything else of the
					operation.
				\item This instruction writes to IO space and reads from
					\texttt{dA}.
				%--------
				\end{itemize}
			\item \texttt{outs64 dA, dB, dC}
				\begin{itemize}
				%--------
				\item This instruction sets the type of \texttt{dA} to
					\texttt{s64} before performing anything else of the
					operation.
				\item This instruction writes to IO space and reads from
					\texttt{dA}.
				%--------
				\end{itemize}
			%--------
			\end{enumerate}
		%--------
		\end{itemize}


	%\printbibliography[heading=bibnumbered,title={Bibliography}]

%--------
\end{document}
