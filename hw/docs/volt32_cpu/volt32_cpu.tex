\documentclass{article}

\usepackage{pbox}
\usepackage{graphicx}
\usepackage{float}
\usepackage{fancyvrb}
\usepackage[T1]{fontenc}
\usepackage{lmodern}
\usepackage{setspace}
\usepackage[nottoc]{tocbibind}
\usepackage[font=large]{caption}
\usepackage{framed}
\usepackage{tabularx}
\usepackage{amsmath}
\usepackage{hyperref}
\usepackage{fontspec}
\usepackage[backend=biber,sorting=none]{biblatex}
%%\usepackage[
%%	backend=biber,
%%	style=ieee,
%%	sorting=none
%%]{biblatex}
%\addbibresource{project_refs.bib}

%% Hide section, subsection, and subsubsection numbering
%\renewcommand{\thesection}{}
%\renewcommand{\thesubsection}{}
%\renewcommand{\thesubsubsection}{}

% Alternative form of doing section stuff
\renewcommand{\thesection}{}
\renewcommand{\thesubsection}{\arabic{section}.\arabic{subsection}}
\makeatletter
\def\@seccntformat#1{\csname #1ignore\expandafter\endcsname\csname the#1\endcsname\quad}
\let\sectionignore\@gobbletwo
\let\latex@numberline\numberline
\def\numberline#1{\if\relax#1\relax\else\latex@numberline{#1}\fi}
\makeatother

\makeatletter
\renewcommand\tableofcontents{%
    \@starttoc{toc}%
}
\makeatother

\newcommand{\respacing}{\doublespacing \singlespacing}
\newcommand{\code}[2][1]{\noindent \texttt{\tab[#1] #2} \\}
\newcommand{\skipline}[0]{\texttt{}\\}
\newcommand{\tnl}[0]{\tabularnewline}

\begin{document}
%--------
	\font\titlefont={Times New Roman} at 20pt
	\title{{\titlefont Volt32 CPU}}

	\font\bottomtextfont={Times New Roman} at 12pt
	\date{{\bottomtextfont} \today}
	\author{{\bottomtextfont Andrew Clark}}

	\setmainfont{Times New Roman}
	\setmonofont{Courier New}

	\maketitle
	\pagenumbering{gobble}

	\newpage
	\pagenumbering{arabic}
	%\tableofcontents
	%\newpage

	%\doublespacing

%\section{Abstract}
	%\setcounter{section}{-1}

\section{Table of Contents}
	\tableofcontents
	\newpage

\section{Registers, Main Widths, etc.}
	\begin{itemize}
	%--------
	\item The machine is an implementation of Line Associative Registers
		(LARs). Both instruction LARs (ILARs) and data LARs (DLARs) are
		included in the design. There are a grand total of 128 ILARs and
		128 DLARs, but they are split between the LARs owned by the
		supervisor mode and the LARs owned by the user mode. There are 64
		supervisor mode ILARs, 64 supervisor mode DLARs, 64 user mode
		ILARs, and 64 user mode DLARs.
	\item Addresses are 32-bit.
	\item 8 types of integer operations are supported, specifically
		\texttt{u8}, \texttt{s8},
		\texttt{u16}, \texttt{s16},
		\texttt{u32}, \texttt{s32},
		\texttt{u64}, and \texttt{s64}.
		\begin{itemize}
		%--------
		\item There are some limitations on operations done with the
			\texttt{u64} and \texttt{s64} types. See the instruction
			lists for more details.
		%--------
		\end{itemize}.
	%\item 32-bit and 64-bit IEEE 754 floats are supported; these types are
	%	named \texttt{f32} and \texttt{f64}, respectively.
	%	\begin{itemize}
	%	%--------
	%	\item There are some limitations on operations of the \texttt{f64}
	%		type, specifically that vector and reduce operations of this
	%		type are done in a serial fashion (though perhaps the 64-bit
	%		FPU will be pipelined).
	%	%--------
	%	\end{itemize}
	\item The machine boots in supervisor mode. On that note, the processor
		jumps to address \texttt{0x0} whenever it enters supervisor mode.
	\item ILARs
		\begin{itemize}
		%--------
		\item In user mode, ILARs 0 to 63 are referred to as \texttt{i0},
			\texttt{i1}, \texttt{i2}, ..., \texttt{i61}, \texttt{i62},
			\texttt{ipc}.
		\item In supervisor mode, ILARs 64 to 127 are referred to as
			\texttt{i0}, \texttt{i1}, \texttt{i2}, ..., \texttt{i61},
			\texttt{i62}, \texttt{ipc}. Note that supervisor mode ILARs are
			encoded into instructions without the most significant bit,
			i.e. supervisor mode \texttt{i0} is encoded as
			\texttt{0b000000}.
		\item The two ILARs called "\texttt{i0}" have all their
			fields set to zero, and when written to, the contents of the
			ILAR does not change.
		\item The two ILARs called "\texttt{ipc}" are the program counters
			for the two operating modes of the processor. The location
			within the \texttt{ipc} ILARs can be computed by taking the low
			6 bits (6 because 128 bytes long ILARs) of their addresses.
		\item An ILAR's data field is 128 bytes long. It is composed of
			32-bit instructions aligned to 32 bits.
		\item An ILAR's scalar offset field is (7 - 2 = 5)-bit due to
			instructions being 32-bit and the data field being 128 bytes
			long.
		\item The base address field of an ILAR is (32 - 6 = 26)-bit.
		\item An ILAR's tag field is 7-bit because there are 128 total
			ILARs.
		%--------
		\end{itemize}
	\item DLARs
		\begin{itemize}
		%--------
		\item In user mode, DLARs 0 to 63 are referred to as 
			\texttt{d0}, \texttt{d1}, \texttt{d2}, ..., \texttt{d57},
			\texttt{d58},
			\texttt{dt0} (aka \texttt{d59}, assembler temporary 0),
			\texttt{dt1} (aka \texttt{d60}, assembler temporary 1),
			%\texttt{dcu8} (aka \texttt{d51},
			%	intended to be used for table of \texttt{u8} constants),
			%\texttt{dcs8} (aka \texttt{d52},
			%	intended to be used for table of \texttt{s8} constants),
			%\texttt{dcu16} (aka \texttt{d53},
			%	intended to be used for table of \texttt{u16} constants),
			%\texttt{dcs16} (aka \texttt{d54},
			%	intended to be used for table of \texttt{s16} constants),
			%\texttt{dcu32} (aka \texttt{d55},
			%	intended to be used for table of \texttt{u32} constants),
			%\texttt{dcs32} (aka \texttt{d56},
			%	intended to be used for table of \texttt{s32} constants),
			%\texttt{dcu64} (aka \texttt{d57},
			%	intended to be used for table of \texttt{u64} constants),
			%\texttt{dcs64} (aka \texttt{d58},
			%	intended to be used for table of \texttt{s64} constants),
			%\texttt{dcf32} (aka \texttt{d59},
			%	intended to be used for table of \texttt{f32} constants),
			%\texttt{dcf64} (aka \texttt{d60},
			%	intended to be used for table of \texttt{f64} constants),
			\texttt{dcp} (aka \texttt{d61},
				intended to be used as the constant pools pointer),
			\texttt{dfp} (aka \texttt{d62},
				intended to be used as the frame pointer),
			\texttt{dsp} (aka \texttt{d63},
				intended to be used as the stack pointer).
		\item In supervisor mode, DLARs 64 to 127 are referred to as
			\texttt{d0}, \texttt{d1}, \texttt{d2}, ..., \texttt{d57},
			\texttt{d58},
			\texttt{dt0} (aka \texttt{d59}, assembler temporary 0),
			\texttt{dt1} (aka \texttt{d60}, assembler temporary 1),
			%\texttt{dcu8} (aka \texttt{d51},
			%	intended to be used for table of \texttt{u8} constants),
			%\texttt{dcs8} (aka \texttt{d52},
			%	intended to be used for table of \texttt{s8} constants),
			%\texttt{dcu16} (aka \texttt{d53},
			%	intended to be used for table of \texttt{u16} constants),
			%\texttt{dcs16} (aka \texttt{d54},
			%	intended to be used for table of \texttt{s16} constants),
			%\texttt{dcu32} (aka \texttt{d55},
			%	intended to be used for table of \texttt{u32} constants),
			%\texttt{dcs32} (aka \texttt{d56},
			%	intended to be used for table of \texttt{s32} constants),
			%\texttt{dcu64} (aka \texttt{d57},
			%	intended to be used for table of \texttt{u64} constants),
			%\texttt{dcs64} (aka \texttt{d58},
			%	intended to be used for table of \texttt{s64} constants),
			%\texttt{dcf32} (aka \texttt{d59},
			%	intended to be used for table of \texttt{f32} constants),
			%\texttt{dcf64} (aka \texttt{d60},
			%	intended to be used for table of \texttt{f64} constants),
			\texttt{dcp} (aka \texttt{d61},
				intended to be used as the constant pools pointer),
			\texttt{dfp} (aka \texttt{d62},
				intended to be used as the frame pointer),
			\texttt{dsp} (aka \texttt{d63},
				intended to be used as the stack pointer).
			Note that supervisor mode DLARs are encoded into instructions
			without the most significant bit, i.e. supervisor mode
			\texttt{d0} is encoded as \texttt{0b000000}.
		\item The two DLARs called "\texttt{d0}" have all their fields set
			to zero, and when written to, the contents of the particular
			\texttt{d0} DLAR do not change.
		\item A DLAR's data field is 128 bytes long. It is composed of the
			scalar data elements of the 128 byte vectors, where the type of
			the scalar data elements is determined by the type tag field of
			the DLAR.
		\item A DLAR's scalar offset field is 7-bit due to the data field
			being 128 bytes long. However, the address of the scalar data
			is forcibly aligned to the width of the DLAR's type when used
			as source operands in instructions that read from the full
			\texttt{address} field of the instruction.
		\item The base address field of a DLAR is (32 - 6 = 26)-bit.
		\item Note that the full address of a scalar located in a DLAR,
			\texttt{dA}, is determined by the following formula:
			\texttt{cat(dA.base\_address, dA.scalar\_offset)}. For types
			other than \texttt{u8} and \texttt{s8}, the  
		\item DLARs can take on the following types (3-bit enum):
			\begin{itemize}
			%--------
			\item 8-bit, unsigned (\texttt{u8})
			\item 8-bit, signed (\texttt{s8})
			\item 16-bit, unsigned (\texttt{u16})
			\item 16-bit, signed (\texttt{s16})
			\item 32-bit, unsigned (\texttt{u32})
			\item 32-bit, signed (\texttt{s32})
			\item 64-bit, unsigned (\texttt{u64});
				only usable for some operations
			\item 64-bit, signed (\texttt{s64});
				only usable for some operations
			%--------
			\end{itemize}
		\item A DLAR's tag field is 7-bit because there are 128 total
			DLARs.
		\item Similarly, a DLAR's reference count field is 7-bit because
			there are 128 total DLARs.
		\item A DLAR's dirty field is 1-bit.
		%--------
		\end{itemize}
	\item The \texttt{ie} register
		\begin{itemize}
		%--------
		\item "\texttt{ie}" is short for "interrupt enable".
		\item This register is 1-bit.
		\item This register is a flag indicating whether or not an
			interrupt can be serviced. It can be read from/written to using
			\texttt{cpy} instructions.
		\item The \texttt{reti} instruction sets \texttt{ie} to
			\texttt{0b1} and returns to user mode from supervisor mode.
		%--------
		\end{itemize}
	\item The \texttt{xct} register
		\begin{itemize}
		%--------
		\item "exception type"
		\item This register is 32-bit.
		\item This register is set to the numerical value of an exception's
			type upon the machine entering supervisor mode. It can be read
			from/written to using \texttt{cpy} instructions.
		%--------
		\end{itemize}
	\item The \texttt{swiarg0} register
		\begin{itemize}
		%--------
		\item This register is 128 bytes long.
		\item This register indicates argument 0 to \texttt{swi}. In
			supervisor mode, it can be read from/written to with
			\texttt{cpy} instructions.
		%--------
		\end{itemize}
	\item The \texttt{swiarg1} register
		\begin{itemize}
		%--------
		\item This register is 128 bytes long.
		\item This register indicates argument 1 to \texttt{swi}. In
			supervisor mode, it can be read from/written to with
			\texttt{cpy} instructions.
		%--------
		\end{itemize}
	\item The \texttt{swiarg2} register
		\begin{itemize}
		%--------
		\item This register is 128 bytes long.
		\item This register indicates argument 2 to \texttt{swi}. In
			supervisor mode, it can be read from/written to with
			\texttt{cpy} instructions.
		%--------
		\end{itemize}
	\item The \texttt{swiarg3} register
		\begin{itemize}
		%--------
		\item This register is 128 bytes long.
		\item This register indicates argument 3 to \texttt{swi}. In
			supervisor mode, it can be read from/written to with
			\texttt{cpy} instructions.
		%--------
		\end{itemize}
	%--------
	\end{itemize}
	\newpage
\section{Exceptions}
	Some instructions may cause an exception to occur, putting the
	processor in supervisor mode.

	The following exceptions may occur during normal execution of a
	program. \texttt{xct} is set to a numerical value representing these
	upon the processor encountering an exception. 

	\begin{itemize}
	%--------
	\item When \texttt{xct == 0x0}:
		Taking a non-software interrupt (which would also set \texttt{ie}
		to \texttt{0b0}).
	\item When \texttt{xct == 0x1}: Division by zero.
	\item When \texttt{xct == 0x2}: Undefined instruction.
	\item When \texttt{xct == 0x3}:
		Instructions where 64-bit ops are not defined.
	\item When \texttt{xct == 0x4}: \texttt{swi}.
	\item When \texttt{xct == 0x5}: \texttt{reti} when in user mode.
	\item When \texttt{xct == 0x6}: \texttt{retx} when in user mode.

	\item When \texttt{xct == 0x7}:
		\texttt{cpy} that reads from \texttt{ie} in user mode.
	\item When \texttt{xct == 0x8}:
		\texttt{cpy} that writes to \texttt{ie} in user mode.
	\item When \texttt{xct == 0x9}:
		\texttt{cpy} that reads from \texttt{xct} in user mode.
	\item When \texttt{xct == 0xa}:
		\texttt{cpy} that writes to \texttt{xct} in user mode.

	\item When \texttt{xct == 0xb}:
		\texttt{cpy} that reads from \texttt{swiarg0} when in user mode.
	\item When \texttt{xct == 0xc}:
		\texttt{cpy} that writes to \texttt{swiarg0} when in user mode.
	\item When \texttt{xct == 0xd}:
		\texttt{cpy} that reads from \texttt{swiarg1} when in user mode.
	\item When \texttt{xct == 0xe}:
		\texttt{cpy} that writes to \texttt{swiarg1} when in user mode.

	\item When \texttt{xct == 0xf}:
		\texttt{cpy} that reads from \texttt{swiarg2} when in user mode.
	\item When \texttt{xct == 0x10}:
		\texttt{cpy} that writes to \texttt{swiarg2} when in user mode.
	\item When \texttt{xct == 0x11}:
		\texttt{cpy} that reads from \texttt{swiarg3} when in user mode.
	\item When \texttt{xct == 0x12}:
		\texttt{cpy} that writes to \texttt{swiarg3} when in user mode.

	\item When \texttt{xct == 0x13}:
		Instructions that read from supervisor mode ILARs or DLARs when in
		user mode.
	\item When \texttt{xct == 0x14}:
		Instructions that write to supervisor mode ILARs or DLARs when in
		user mode.
	\item When \texttt{xct == 0x15}:
		When user mode \texttt{ipc}'s next destination is not in any ILARs.
	%--------
	\end{itemize}
	\newpage
\section{Instructions}
	\subsection{Group 0 Instructions}
		\begin{itemize}
		%--------
		\item Encoding 0: \texttt{0000 aaaa aabb bbbb  cccc cc00 000v oooo}
		\item Encoding 1: \texttt{0000 aaaa aabb bbbb  cccc ccdd dddd oooo}
			\begin{itemize}
			%--------
			\item \texttt{a}: DLAR a
			\item \texttt{b}: DLAR b or ILAR b
			\item \texttt{c}: DLAR c
			\item \texttt{d}: DLAR d
			\item \texttt{v}:
				\begin{itemize}
				%--------
				\item when \texttt{0b0}: scalar operation. The assembly
					syntax indicating a scalar operation simply adds
					"\texttt{.s}" to the instruction's name.
				\item when \texttt{0b1}: vector operation. The assembly
					syntax indicating a vector operation simply adds
					"\texttt{.v}" to the instruction's name.
				%--------
				\end{itemize}
			\item \texttt{o}: Opcode
			%--------
			\end{itemize}
		\item Most instructions in this group use Encoding 0.
			The instructions \texttt{div.s} and \texttt{div.v} use Encoding
			1. These two instructions have opcodes of \texttt{0b1110} and
			\texttt{0b1111}, respectively, such that bit 0 of the
			instruction specifies the same information as the \texttt{v}
			bit does for Encoding 0.
		\item Instruction List:
			\begin{enumerate}
			%--------
			\item \texttt{add dA, dB, dC}
				%\begin{itemize}
				%%--------
				%\item This instruction causes an exception if \texttt{dA}
				%	is of the following types: \texttt{u64}, \texttt{s64}.
				%%--------
				%\end{itemize}
			\item \texttt{sub dA, dB, dC}
				%\begin{itemize}
				%%--------
				%\item This instruction causes an exception if \texttt{dA}
				%	is of the following types: \texttt{u64}, \texttt{s64}.
				%%--------
				%\end{itemize}
			\item \texttt{slt dA, dB, dC}
				%\begin{itemize}
				%%--------
				%\item This instruction causes an exception if \texttt{dA}
				%	is of the following types: \texttt{u64}, \texttt{s64}.
				%%--------
				%\end{itemize}
			\item \texttt{mul dA, dB, dC}
				\begin{itemize}
				%--------
				\item This instruction causes an exception if \texttt{dB}
					or \texttt{dC} is of the following types: \texttt{u64},
					\texttt{s64}.
				%--------
				\end{itemize}

			\item \texttt{and dA, dB, dC}
			\item \texttt{or dA, dB, dC}
			\item \texttt{xor dA, dB, dC}
			\item \texttt{shl dA, dB, dC}
				\begin{itemize}
				%--------
				\item Logical shift left.
				\item This instruction causes an exception if \texttt{dA},
					\texttt{dB}, or \texttt{dC} is of the following types:
					\texttt{u64}, \texttt{s64}.
				\item This instruction casts a temporary copy of
					\texttt{dC} to the unsigned type that is the same size
					as \texttt{dA}'s type and uses that instead of
					\texttt{dC}.
				%--------
				\end{itemize}

			\item \texttt{shr dA, dB, dC}
				\begin{itemize}
				%--------
				\item Logical shift right if \texttt{dA} is unsigned, or
					arithmetic shift right if \texttt{dA} is signed.
				\item This instruction causes an exception if \texttt{dA},
					\texttt{dB}, or \texttt{dC} is of the following types:
					\texttt{u64}, \texttt{s64}.
				\item This instruction casts a temporary copy of
					\texttt{dC} to the unsigned type that is the same size
					as \texttt{dA}'s type and uses that instead of
					\texttt{dC}.
				%--------
				\end{itemize}
			\item \texttt{rol dA, dB, dC}
				\begin{itemize}
				%--------
				\item Rotate left.
				\item This instruction causes an exception if \texttt{dA},
					\texttt{dB}, or \texttt{dC} is of the following types:
					\texttt{u64}, \texttt{s64}.
				\item This instruction casts a temporary copy of
					\texttt{dC} to the unsigned type that is the same size
					as \texttt{dA}'s type and uses that instead of
					\texttt{dC}.
				%--------
				\end{itemize}
			\item \texttt{ror dA, dB, dC}
				\begin{itemize}
				%--------
				\item Rotate right.
				\item This instruction causes an exception if \texttt{dA},
					\texttt{dB}, or \texttt{dC} is of the following types:
					\texttt{u64}, \texttt{s64}.
				\item This instruction casts a temporary copy of
					\texttt{dC} to the unsigned type that is the same size
					as \texttt{dA}'s type and uses that instead of
					\texttt{dC}.
				%--------
				\end{itemize}
			\item \texttt{add dA, dB.addr, dC}
				\begin{itemize}
				%--------
				\item This instruction causes an exception if \texttt{dA}
					or \texttt{dC} is of the following types: \texttt{u64},
					\texttt{s64}.
				\item For \texttt{add.s dA, dB.addr, dC}, this instruction
					uses the value of
					\texttt{cast(dA.type, align(dB.type, dB.addr))}
					as the second argument of the \texttt{add}.
				\item For \texttt{add.v dA, dB.addr, dC}, this instruction
					duplicates the value of
					\texttt{cast(dA.type, align(dB.type, dB.addr))}
					(a scalar) into a temporary (DLAR data's length number
					of bytes long) vector of element type \texttt{dA.type}
					for the purposes of this calculation.
				%--------
				\end{itemize}

			\item \texttt{shl dA, dB.addr, dC}
				\begin{itemize}
				%--------
				\item This instruction causes an exception if \texttt{dA}
					or \texttt{dC} is of the following types: \texttt{u64},
					\texttt{s64}.
				\item For \texttt{shl.s dA, dB.addr, dC}, this instruction
					uses the value of
					\texttt{cast(dA.type, align(dB.type, dB.addr))}
					as the second argument of the \texttt{shl}.
				\item For \texttt{shl.v dA, dB.addr, dC}, this instruction
					duplicates the value of
					\texttt{cast(dA.type, align(dB.type, dB.addr))}
					(a scalar) into a temporary (DLAR data's length number
					of bytes long) vector of element type \texttt{dA.type}
					for the purposes of this calculation.
				\item This instruction casts a temporary copy of
					\texttt{dC} to the unsigned type that is the same size
					as \texttt{dA}'s type and uses that instead of
					\texttt{dC}.
				%--------
				\end{itemize}
			\item \texttt{add dA, iB.addr, dC}
				\begin{itemize}
				%--------
				%\item This instruction causes an exception if \texttt{dA}
				%	or \texttt{dC} is of the following types: \texttt{u64},
				%	\texttt{s64}.
				\item For \texttt{add.s dA, iB.addr, dC}, this instruction
					uses the value of
					\texttt{cast(dA.type, align(u32, dB.addr))}
					as the second argument of the \texttt{add}.
				\item For \texttt{add.v dA, iB.addr, dC}, this instruction
					duplicates the value of
					\texttt{cast(dA.type, align(u32, iB.addr))}
					(a scalar) into a temporary (DLAR data's length number
					of bytes long) vector of element type \texttt{dA.type}
					for the purposes of this calculation.
				%--------
				\end{itemize}
			\item \texttt{div.s dA, dB, dC, dD}
				\begin{itemize}
				%--------
				%\item This instruction causes an exception if \texttt{dB}
				%	or \texttt{dC} is of the following types:
				%	\texttt{u64}, \texttt{s64}.
				\item This instruction causes an exception if \texttt{dC}
					is of the following types: \texttt{u64}, \texttt{s64}.
				\item This instruction writes the quotient into
					\texttt{dA}, and the remainder into \texttt{dD}.
				%--------
				\end{itemize}
			\item \texttt{div.v dA, dB, dC, dD}
				\begin{itemize}
				%--------
				%\item This instruction causes an exception if \texttt{dB}
				%	or \texttt{dC} is of the following types:
				%	\texttt{u64}, \texttt{s64}.
				\item This instruction causes an exception if \texttt{dC}
					is of the following types: \texttt{u64}, \texttt{s64}.
				\item This instruction writes the quotient into
					\texttt{dA}, and the remainder into \texttt{dD}.
				%--------
				\end{itemize}
			%--------
			\end{enumerate}
		%--------
		\end{itemize}
		\newpage
	\subsection{Group 1 Instructions}
		\begin{itemize}
		%--------
		\item Encoding: \texttt{0001 aaaa aabb bbbb  cccc cc00 0000 0ooo}
			\begin{itemize}
			%--------
			\item \texttt{a}: DLAR a
			\item \texttt{b}: DLAR b
			\item \texttt{c}: DLAR c
			%\item \texttt{i}: \texttt{simm12} (sign-extended 12-bit
			%	immediate)
			\item \texttt{o}: Opcode
			%--------
			\end{itemize}
		%\item For the first 10 instructions with names of the form
		%	"\texttt{ld<type>}", the value of 
		%	\texttt{cast(u32, dB.scalar\_data)}
		%	is added to the value of
		%	\texttt{cast(u32, dC.scalar\_data)}
		%	to calculate the address being loaded from. Also, for these
		%	same instructions, the type \texttt{dA} is set to is indicated
		%	in the instruction name, with, for example, \texttt{ldu8}
		%	setting \texttt{dA}'s type to \texttt{u8}.
		%\item For the second 10 instructions with names of the form
		%	"\texttt{ld<type>}", the value of
		%	\texttt{cast(u32, align(dB.type, dB.addr))}
		%	is added to the
		%	\texttt{cast(u32, dC.scalar\_data)}
		%	to calculate the address being loaded from.  Also, for these
		%	same instructions, the type \texttt{dA} is set to is indicated
		%	in the instruction name, with, for example, \texttt{lds32}
		%	setting \texttt{dA}'s type to \texttt{s32}.
		%\item For the instructions with names of the form
		%	"\texttt{st<type>}", the value of
		%	\texttt{cast(u32, dB.scalar\_data)}
		%	is added to the value of
		%	\texttt{cast(u32, dC.scalar\_data)}
		%	to calculate the address being stored to. Also, for these same
		%	instructions, the type \texttt{dA} is set to is indicated in
		%	the instruction name, with, for example, \texttt{sts64} setting
		%	\texttt{dA}'s type to \texttt{s64}.
		\item Instruction List:
			\begin{enumerate}
			%--------
			\item \texttt{add.r dA, dB}
				\begin{itemize}
				%--------
				\item This instruction casts a temporary copy of
					\texttt{dB} to the \texttt{dA}'s type and performs a
					sum of all the scalar data elements of the temporary
					copy of \texttt{dB}, then stores the result in
					\texttt{dA}'s scalar data.
				%\item This instruction causes an exception if \texttt{dA}
				%	is of the following types: \texttt{u64}, \texttt{s64}.
				%--------
				\end{itemize}
			\item \texttt{mul.r dA, dB}
				\begin{itemize}
				%--------
				\item This instruction casts a temporary copy of
					\texttt{dB} to the \texttt{dA}'s type and performs a
					product of all the scalar data elements of the
					temporary copy of \texttt{dB}, then stores the result
					in \texttt{dA}'s scalar data.
				\item This instruction causes an exception if \texttt{dA}
					is of the following types: \texttt{u64}, \texttt{s64}.
				%--------
				\end{itemize}
			\item \texttt{max.r dA, dB}
				\begin{itemize}
				%--------
				\item This instruction casts a temporary copy of
					\texttt{dB} to the \texttt{dA}'s type and finds the
					scalar data element of the temporary copy of
					\texttt{dB} that is the maximum, then stores the result
					in \texttt{dA}'s scalar data.
				%\item This instruction causes an exception if \texttt{dA}
				%	is of the following types: \texttt{u64}, \texttt{s64}.
				%--------
				\end{itemize}
			\item \texttt{min.r dA, dB}
				\begin{itemize}
				%--------
				\item This instruction casts a temporary copy of
					\texttt{dB} to the \texttt{dA}'s type and finds the
					scalar data element of the temporary copy of
					\texttt{dB} that is the minimum, then stores the result
					in \texttt{dA}'s scalar data.
				%\item This instruction causes an exception if \texttt{dA}
				%	is of the following types: \texttt{u64}, \texttt{s64}.
				%--------
				\end{itemize}

			\item \texttt{and.r dA, dB}
				\begin{itemize}
				%--------
				\item This instruction casts a temporary copy of
					\texttt{dB} to the \texttt{dA}'s type and performs a
					bitwise AND reduction of all the scalar data elements
					of the temporary copy of \texttt{dB}, then stores the
					result in \texttt{dA}'s scalar data.
				%\item This instruction causes an exception if \texttt{dA}
				%	is of the following types: \texttt{u64}, \texttt{s64}.
				%--------
				\end{itemize}
			\item \texttt{or.r dA, dB}
				\begin{itemize}
				%--------
				\item This instruction casts a temporary copy of
					\texttt{dB} to the \texttt{dA}'s type and performs a
					bitwise OR reduction of all the scalar data elements of
					the temporary copy of \texttt{dB}, then stores the
					result in \texttt{dA}'s scalar data.
				%\item This instruction causes an exception if \texttt{dA}
				%	is of the following types: \texttt{u64}, \texttt{s64}.
				%--------
				\end{itemize}
			\item \texttt{xor.r dA, dB}
				\begin{itemize}
				%--------
				\item This instruction casts a temporary copy of
					\texttt{dB} to the \texttt{dA}'s type and performs a
					bitwise XOR reduction of all the scalar data elements
					of the temporary copy of \texttt{dB}, then stores the
					result in \texttt{dA}'s scalar data.
				%\item This instruction causes an exception if \texttt{dA}
				%	is of the following types: \texttt{u64}, \texttt{s64}.
				%--------
				\end{itemize}
			\item \textit{Reserved for future expansion.}

			%\item \textit{Reserved for future expansion.}
			%\item \textit{Reserved for future expansion.}
			%\item \textit{Reserved for future expansion.}
			%\item \textit{Reserved for future expansion.}

			%\item \textit{Reserved for future expansion.}
			%\item \textit{Reserved for future expansion.}
			%\item \textit{Reserved for future expansion.}
			%\item \textit{Reserved for future expansion.}
			%%--------
			%\item \texttt{ldu8 dA, dB, dC}
			%\item \texttt{lds8 dA, dB, dC}
			%\item \texttt{ldu16 dA, dB, dC}
			%\item \texttt{lds16 dA, dB, dC}

			%\item \texttt{ldu32 dA, dB, dC}
			%\item \texttt{lds32 dA, dB, dC}
			%\item \texttt{ldu64 dA, dB, dC}
			%\item \texttt{lds64 dA, dB, dC}

			%%\item \texttt{ldf32 dA, dB, dC}
			%%\item \texttt{ldf64 dA, dB, dC}
			%\item \textit{Reserved for future expansion.}
			%\item \textit{Reserved for future expansion.}
			%\item \textit{Reserved for future expansion.}
			%\item \textit{Reserved for future expansion.}

			%\item \textit{Reserved for future expansion.}
			%\item \textit{Reserved for future expansion.}
			%\item \textit{Reserved for future expansion.}
			%\item \textit{Reserved for future expansion.}
			%%--------
			%\item \texttt{ldu8 dA, dB.addr, dC}
			%\item \texttt{lds8 dA, dB.addr, dC}
			%\item \texttt{ldu16 dA, dB.addr, dC}
			%\item \texttt{lds16 dA, dB.addr, dC}

			%\item \texttt{ldu32 dA, dB.addr, dC}
			%\item \texttt{lds32 dA, dB.addr, dC}
			%\item \texttt{ldu64 dA, dB.addr, dC}
			%\item \texttt{lds64 dA, dB.addr, dC}

			%%\item \texttt{ldf32 dA, dB.addr, dC}
			%%\item \texttt{ldf64 dA, dB.addr, dC}
			%\item \textit{Reserved for future expansion.}
			%\item \textit{Reserved for future expansion.}
			%\item \textit{Reserved for future expansion.}
			%\item \textit{Reserved for future expansion.}

			%\item \textit{Reserved for future expansion.}
			%\item \textit{Reserved for future expansion.}
			%\item \textit{Reserved for future expansion.}
			%\item \textit{Reserved for future expansion.}
			%%--------
			%\item \texttt{stu8 dA, dB, dC}
			%\item \texttt{sts8 dA, dB, dC}
			%\item \texttt{stu16 dA, dB, dC}
			%\item \texttt{sts16 dA, dB, dC}

			%\item \texttt{stu32 dA, dB, dC}
			%\item \texttt{sts32 dA, dB, dC}
			%\item \texttt{stu64 dA, dB, dC}
			%\item \texttt{sts64 dA, dB, dC}

			%%\item \texttt{stf32 dA, dB, dC}
			%%\item \texttt{stf64 dA, dB, dC}
			%\item \textit{Reserved for future expansion.}
			%\item \textit{Reserved for future expansion.}
			%\item \textit{Reserved for future expansion.}
			%\item \textit{Reserved for future expansion.}

			%\item \textit{Reserved for future expansion.}
			%\item \textit{Reserved for future expansion.}
			%\item \textit{Reserved for future expansion.}
			%\item \textit{Reserved for future expansion.}
			%--------
			\end{enumerate}
		%--------
		\end{itemize}
		\newpage

	\subsection{Group 2 Instructions}
		\begin{itemize}
		%--------
		%\item Encoding: \texttt{0010 aaaa aabb bbbb  cccc ccii iiii iooo}
		\item Encoding: \texttt{0010 aaaa aabb bbbb  iiii iiii iiii oooo}
			\begin{itemize}
			%--------
			\item \texttt{a}: DLAR a
			\item \texttt{b}: DLAR b
			%\item \texttt{c}: DLAR c
			%\item \texttt{i}: \texttt{simm7} (sign-extended 7-bit
			%	immediate)
			\item \texttt{i}: \texttt{simm12} (sign-extended 12-bit
				immediate)
			\item \texttt{o}: Opcode
			%--------
			\end{itemize}
		%\item For these instructions, the \texttt{dB} DLAR's scalar
		%	data field (temporarily casted to the \texttt{u32} type) and
		%	the \texttt{dC} DLAR's address field are added together,
		%	plus \texttt{simm7} (sign-extended to 32-bit), to calculate the
		%	address being loaded from.
		%\item Also, the type that \texttt{dA} is set to is indicated in the
		%	instruction name, with, for example, \texttt{ldu8} setting
		%	\texttt{dA}'s type to \texttt{u8}.
		\item For these instructions, the value of
			\texttt{cast(u32, dB.scalar\_data) + cast(s32, simm12)}
			is used as the address to load from.
		\item Also, the type \texttt{dA} is set to is indicated in the
			instruction name, with, for example, \texttt{ldu8i} setting
			\texttt{dA}'s type to \texttt{u8}.
		\item Instruction List:
			\begin{enumerate}
			%--------
			%\item \texttt{ldu8 dA, dB, dC.addr, simm7}
			%\item \texttt{lds8 dA, dB, dC.addr, simm7}
			%\item \texttt{ldu16 dA, dB, dC.addr, simm7}
			%\item \texttt{lds16 dA, dB, dC.addr, simm7}
			%\item \texttt{ldu32 dA, dB, dC.addr, simm7}
			%\item \texttt{lds32 dA, dB, dC.addr, simm7}
			%\item \texttt{ldu64 dA, dB, dC.addr, simm7}
			%\item \texttt{lds64 dA, dB, dC.addr, simm7}

			\item \texttt{ldu8i dA, dB, simm12}
			\item \texttt{lds8i dA, dB, simm12}
			\item \texttt{ldu16i dA, dB, simm12}
			\item \texttt{lds16i dA, dB, simm12}

			\item \texttt{ldu32i dA, dB, simm12}
			\item \texttt{lds32i dA, dB, simm12}
			\item \texttt{ldu64i dA, dB, simm12}
			\item \texttt{lds64i dA, dB, simm12}

			%\item \texttt{ldf32 dA, dB, simm12}
			%\item \texttt{ldf64 dA, dB, simm12}
			\item \textit{Reserved for future expansion.}
			\item \textit{Reserved for future expansion.}
			\item \textit{Reserved for future expansion.}
			\item \textit{Reserved for future expansion.}

			\item \textit{Reserved for future expansion.}
			\item \textit{Reserved for future expansion.}
			\item \textit{Reserved for future expansion.}
			\item \textit{Reserved for future expansion.}
			%--------
			\end{enumerate}
		%--------
		\end{itemize}
		\newpage
	\subsection{Group 3 Instructions}
		\begin{itemize}
		%--------
		\item Encoding: \texttt{0011 aaaa aabb bbbb  iiii iiii iiii oooo}
			\begin{itemize}
			%--------
			\item \texttt{a}: DLAR a
			\item \texttt{b}: DLAR b
			\item \texttt{i}: \texttt{simm12} (sign-extended 12-bit
				immediate)
			\item \texttt{o}: Opcode
			%--------
			\end{itemize}
		\item For these instructions, the value of
			\texttt{cast(u32, align(dB.type, dB.addr)) + cast(s32, simm12)}
			is used as the address to load from.
		\item Also, the type \texttt{dA} is set to is indicated in the
			instruction name, with, for example, \texttt{ldu8} setting
			\texttt{dA}'s type to \texttt{u8}.
		\item Instruction List:
			\begin{enumerate}
			%--------
			\item \texttt{ldu8i dA, dB.addr, simm12}
			\item \texttt{lds8i dA, dB.addr, simm12}
			\item \texttt{ldu16i dA, dB.addr, simm12}
			\item \texttt{lds16i dA, dB.addr, simm12}

			\item \texttt{ldu32i dA, dB.addr, simm12}
			\item \texttt{lds32i dA, dB.addr, simm12}
			\item \texttt{ldu64i dA, dB.addr, simm12}
			\item \texttt{lds64i dA, dB.addr, simm12}

			%\item \texttt{ldf32 dA, dB.addr, simm12}
			%\item \texttt{ldf64 dA, dB.addr, simm12}
			\item \textit{Reserved for future expansion.}
			\item \textit{Reserved for future expansion.}
			\item \textit{Reserved for future expansion.}
			\item \textit{Reserved for future expansion.}

			\item \textit{Reserved for future expansion.}
			\item \textit{Reserved for future expansion.}
			\item \textit{Reserved for future expansion.}
			\item \textit{Reserved for future expansion.}
			%--------
			\end{enumerate}
		%--------
		\end{itemize}
		\newpage

	\subsection{Group 4 Instructions}
		\begin{itemize}
		%--------
		\item Encoding: \texttt{0100 aaaa aabb bbbb  cccc cc00 0000 oooo}
			\begin{itemize}
			%--------
			\item \texttt{a}: DLAR a
			\item \texttt{b}: DLAR b
			\item \texttt{c}: DLAR c
			%\item \texttt{d}: DLAR d
			\item \texttt{o}: Opcode
			%--------
			\end{itemize}
		\item For these instructions, the value of
			\texttt{cast(u32, dB.scalar\_data)
				+ cast(u32, dC.scalar\_data)}
			is used as the address to load from.
		\item Also, the type \texttt{dA} is set to is indicated in the
			instruction name, with, for example, \texttt{ldu8} setting
			\texttt{dA}'s type to \texttt{u8}.
		\item Instruction List:
			\begin{enumerate}
			%--------
			\item \texttt{ldu8 dA, dB, dC}
			\item \texttt{lds8 dA, dB, dC}
			\item \texttt{ldu16 dA, dB, dC}
			\item \texttt{lds16 dA, dB, dC}

			\item \texttt{ldu32 dA, dB, dC}
			\item \texttt{lds32 dA, dB, dC}
			\item \texttt{ldu64 dA, dB, dC}
			\item \texttt{lds64 dA, dB, dC}

			%\item \texttt{ldf32 dA, dB, dC}
			%\item \texttt{ldf64 dA, dB, dC}
			\item \textit{Reserved for future expansion.}
			\item \textit{Reserved for future expansion.}
			\item \textit{Reserved for future expansion.}
			\item \textit{Reserved for future expansion.}

			\item \textit{Reserved for future expansion.}
			\item \textit{Reserved for future expansion.}
			\item \textit{Reserved for future expansion.}
			\item \textit{Reserved for future expansion.}
			%--------
			\end{enumerate}
		%--------
		\end{itemize}
		\newpage


	\subsection{Group 5 Instructions}
		\begin{itemize}
		%--------
		%\item Encoding: \texttt{0101 aaaa aabb bbbb  cccc ccii iiii iooo}
		\item Encoding: \texttt{0101 aaaa aabb bbbb  iiii iiii iiii oooo}
			\begin{itemize}
			%--------
			\item \texttt{a}: DLAR a
			\item \texttt{b}: DLAR b
			%\item \texttt{c}: DLAR c
			%\item \texttt{i}: \texttt{simm7} (sign-extended 7-bit
			%	immediate)
			\item \texttt{i}: \texttt{simm12} (sign-extended 12-bit
				immediate)
			\item \texttt{o}: Opcode
			%--------
			\end{itemize}
		%\item For these instructions, the \texttt{dB} DLAR's scalar
		%	data field (temporarily casted to the \texttt{u32} type) and
		%	the \texttt{dC} DLAR's address field are added together,
		%	plus \texttt{simm7} (sign-extended to 32-bit), to calculate the
		%	address being stored to.
		%\item Also, the type that \texttt{dA} is set to is indicated in the
		%	instruction name, with, for example, \texttt{stu8} setting
		%	\texttt{dA}'s type to \texttt{u8}.
		\item For these instructions, the value of
			\texttt{cast(u32, dB.scalar\_data) + cast(s32, simm12)}
			is used as the address being stored to.
		\item Also, the type \texttt{dA} is set to is indicated in the
			instruction name, with, for example, \texttt{stu8} setting
			\texttt{dA}'s type to \texttt{u8}.
		\item Instruction List:
			\begin{enumerate}
			%--------
			%\item \texttt{stu8 dA, dB, dC.addr, simm7}
			%\item \texttt{sts8 dA, dB, dC.addr, simm7}
			%\item \texttt{stu16 dA, dB, dC.addr, simm7}
			%\item \texttt{sts16 dA, dB, dC.addr, simm7}
			%\item \texttt{stu32 dA, dB, dC.addr, simm7}
			%\item \texttt{sts32 dA, dB, dC.addr, simm7}
			%\item \texttt{stu64 dA, dB, dC.addr, simm7}
			%\item \texttt{sts64 dA, dB, dC.addr, simm7}

			\item \texttt{stu8 dA, dB, simm12}
			\item \texttt{sts8 dA, dB, simm12}
			\item \texttt{stu16 dA, dB, simm12}
			\item \texttt{sts16 dA, dB, simm12}

			\item \texttt{stu32 dA, dB, simm12}
			\item \texttt{sts32 dA, dB, simm12}
			\item \texttt{stu64 dA, dB, simm12}
			\item \texttt{sts64 dA, dB, simm12}

			%\item \texttt{stf32 dA, dB, simm12}
			%\item \texttt{stf64 dA, dB, simm12}
			\item \textit{Reserved for future expansion.}
			\item \textit{Reserved for future expansion.}
			\item \textit{Reserved for future expansion.}
			\item \textit{Reserved for future expansion.}

			\item \textit{Reserved for future expansion.}
			\item \textit{Reserved for future expansion.}
			\item \textit{Reserved for future expansion.}
			\item \textit{Reserved for future expansion.}
			%--------
			\end{enumerate}
		%--------
		\end{itemize}
		\newpage

	\subsection{Group 6 Instructions}
		\begin{itemize}
		%--------
		\item Encoding: \texttt{0110 aaaa aabb bbbb  0000 0000 0000 oooo}
			\begin{itemize}
			%--------
			\item \texttt{a}: DLAR a
			\item \texttt{b}: DLAR b
			\item \texttt{o}: Opcode
			%--------
			\end{itemize}
		\item For these instructions, the \texttt{dB} register's scalar
			data field is used.
		\item Instruction List:
			\begin{enumerate}
			%--------
			\item \texttt{dpu8 dA, dB}
				\begin{itemize}
				%--------
				\item This instruction casts (a temporary copy of) the
					scalar data of \texttt{dB} to the \texttt{u8} type. The
					casted scalar data is then stored into every
					\texttt{u8} vector element of \texttt{dA}. The type of
					\texttt{dA} is then changed to \texttt{u8}.
				%--------
				\end{itemize}
			\item \texttt{dps8 dA, dB}
				\begin{itemize}
				%--------
				\item This instruction casts (a temporary copy of) the
					scalar data of \texttt{dB} to the \texttt{s8} type. The
					casted scalar data is then stored into every
					\texttt{s8} vector element of \texttt{dA}. The type of
					\texttt{dA} is then changed to \texttt{s8}.
				%--------
				\end{itemize}
			\item \texttt{dpu16 dA, dB}
				\begin{itemize}
				%--------
				\item This instruction casts (a temporary copy of) the
					scalar data of \texttt{dB} to the \texttt{u16} type.
					The casted scalar data is then stored into every
					\texttt{u16} vector element of \texttt{dA}. The type of
					\texttt{dA} is then changed to \texttt{u16}.
				%--------
				\end{itemize}
			\item \texttt{dps16 dA, dB}
				\begin{itemize}
				%--------
				\item This instruction casts (a temporary copy of) the
					scalar data of \texttt{dB} to the \texttt{s16} type.
					The casted scalar data is then stored into every
					\texttt{s16} vector element of \texttt{dA}. The type of
					\texttt{dA} is then changed to \texttt{s16}.
				%--------
				\end{itemize}

			\item \texttt{dpu32 dA, dB}
				\begin{itemize}
				%--------
				\item This instruction casts (a temporary copy of) the
					scalar data of \texttt{dB} to the \texttt{u32} type.
					The casted scalar data is then stored into every
					\texttt{u32} vector element of \texttt{dA}. The type of
					\texttt{dA} is then changed to \texttt{u32}.
				%--------
				\end{itemize}
			\item \texttt{dps32 dA, dB}
				\begin{itemize}
				%--------
				\item This instruction casts (a temporary copy of) the
					scalar data of \texttt{dB} to the \texttt{s32} type.
					The casted scalar data is then stored into every
					\texttt{s32} vector element of \texttt{dA}. The type of
					\texttt{dA} is then changed to \texttt{s32}.
				%--------
				\end{itemize}
			\item \texttt{dpu64 dA, dB}
				\begin{itemize}
				%--------
				\item This instruction casts (a temporary copy of) the
					scalar data of \texttt{dB} to the \texttt{u64} type.
					The casted scalar data is then stored into every
					\texttt{u64} vector element of \texttt{dA}. The type of
					\texttt{dA} is then changed to \texttt{u64}.
				%--------
				\end{itemize}
			\item \texttt{dps64 dA, dB}
				\begin{itemize}
				%--------
				\item This instruction casts (a temporary copy of) the
					scalar data of \texttt{dB} to the \texttt{s64} type.
					The casted scalar data is then stored into every
					\texttt{s64} vector element of \texttt{dA}. The type of
					\texttt{dA} is then changed to \texttt{s64}.
				%--------
				\end{itemize}

			%\item \texttt{dpf32 dA, dB}
			%	\begin{itemize}
			%	%--------
			%	\item This instruction casts (a temporary copy of) the
			%		scalar data of \texttt{dB} to the \texttt{f32} type.
			%		The casted scalar data is then stored into every
			%		\texttt{s64} vector element of \texttt{dA}. The type of
			%		\texttt{dA} is then changed to \texttt{s64}.
			%	%--------
			%	\end{itemize}
			%\item \texttt{dpf64 dA, dB}
			%	\begin{itemize}
			%	%--------
			%	\item This instruction casts (a temporary copy of) the
			%		scalar data of \texttt{dB} to the \texttt{f64} type.
			%		The casted scalar data is then stored into every
			%		\texttt{s64} vector element of \texttt{dA}. The type of
			%		\texttt{dA} is then changed to \texttt{s64}.
			%	%--------
			%	\end{itemize}
			\item \textit{Reserved for future expansion.}
			\item \textit{Reserved for future expansion.}
			\item \textit{Reserved for future expansion.}
			\item \textit{Reserved for future expansion.}

			\item \textit{Reserved for future expansion.}
			\item \textit{Reserved for future expansion.}
			\item \textit{Reserved for future expansion.}
			\item \textit{Reserved for future expansion.}
			%--------
			\end{enumerate}
		%--------
		\end{itemize}
		\newpage

	\subsection{Group 7 Instructions}
		\begin{itemize}
		%--------
		\item Encoding 0: \texttt{0111 aaaa aabb bbbb  cccc cc00 000j jjjo}
		\item Encoding 1: \texttt{0111 aaaa aabb bbbb  iiii iiii iiij jjjo}
			\begin{itemize}
			%--------
			\item \texttt{a}: ILAR a
			\item \texttt{b}: ILAR b
			\item \texttt{c}: DLAR c
			\item \texttt{i}: \texttt{isimm11} (11-bit immediate, left
				shifted by 2, then sign-extended)
			\item \texttt{j}: \texttt{jimm4}, the number of consecutive
				ILARs past \texttt{iA} to \texttt{fetch} into.
			\item \texttt{o}: Opcode
			%--------
			\end{itemize}

		\item For instructions using Encoding 0, the address to
			\texttt{fetch} from is computed by adding the address field of
			\texttt{iB} to the scalar data field (temporarily casted to the
			\texttt{u32} type) of the \texttt{dC} DLAR.
		%\item Also, the address to \texttt{fetch} from is computed by
		%	adding these two temporary values to a temporary value
		%	equivalent to the 32-bit sign extended value of
		%	\texttt{isimm11 << 2}.
		\item For instructions using Encoding 1, the address to
			\texttt{fetch} from is computed by adding the address field of
			\texttt{iB} to the the value
			\texttt{cast(s32, (isimm11 << 2))}.
		\item Additionally, this and any other instructions with
			"\texttt{fetch}" in their names are the only way to fetch
			instructions from memory (or other ILARs) in this CPU's design.
			The instruction pipeline only automatically fetches
			instructions from \texttt{ipc}, with straight line code that
			overflows outside of \texttt{ipc} causing \texttt{ipc} to set
			its data field to that of the other ILARs that have the data
			from the destination. If no ILARs have the data from the
			destination, an exception is thrown.

		\item Instruction List:
			\begin{enumerate}
			%--------
			\item \texttt{fetch iA, iB, dC, jimm4}
				\begin{itemize}
				%--------
				\item This instruction uses Encoding 0.
				%--------
				\end{itemize}
			\item \texttt{fetch iA, iB, isimm11, jimm4}
				\begin{itemize}
				%--------
				\item This instruction uses Encoding 1.
				%--------
				\end{itemize}
			%--------
			\end{enumerate}
		%--------
		\end{itemize}
		\newpage

	\subsection{Group 8 Instructions}
		\begin{itemize}
		%--------
		%\item Encoding: \texttt{1000 aaaa aabb bbbb  cccc ccdd dddv oooo}
		\item Encoding: \texttt{1000 aaaa aabb bbbb  iiii ijjj jj0v oooo}
			\begin{itemize}
			%--------
			\item \texttt{a}: DLAR a
			\item \texttt{b}: ILAR b
			%\item \texttt{c}: DLAR c
			\item \texttt{i}: \texttt{iimm5} (5-bit immediate, left shifted
				by 2, then zero-extended)
			\item \texttt{j}: \texttt{jimm5} (5-bit immediate, left shifted
				by 2, then zero-extended)
			\item \texttt{v}:
				\begin{itemize}
				%--------
				\item when \texttt{0b0}: scalar operation (uses the scalar
					data of \texttt{dA}). The assembly syntax indicating a
					scalar operation simply adds "\texttt{.s}" to the
					instruction's name.
				\item when \texttt{0b1}: vector operation (uses the vector
					data of \texttt{dA}). The assembly syntax indicating a
					vector operation simply adds "\texttt{.v}" to the
					instruction's name.
				%--------
				\end{itemize}
			\item \texttt{o}: Opcode
			%--------
			\end{itemize}
		\item These instructions use the scalar or vector data field of
			\texttt{dA}.
		\item Instruction List:
			\begin{enumerate}
			%--------
			\item \texttt{sel dA, iB, iimm5, jimm5}
				\begin{itemize}
				%--------
				\item This instruction jumps to \texttt{iB[iimm5 << 2]} if
					the particular data field of \texttt{dA} is non-zero,
					otherwise to the address \texttt{iB[jimm5 << 2]}.
				%--------
				\end{itemize}
			\item \texttt{jz dA, iB, iimm5}
				\begin{itemize}
				%--------
				\item This instruction jumps to \texttt{iB[iimm5 << 2]}
					if the particular data field of \texttt{dA} is zero.
				%--------
				\end{itemize}
			\item \texttt{jnz dA, iB, iimm5}
				\begin{itemize}
				%--------
				\item This instruction jumps to \texttt{iB[iimm5 << 2]}
					if the particular data field of \texttt{dA} is
					non-zero.
				%--------
				\end{itemize}
			\item \texttt{reti dA}
				\begin{itemize}
				%--------
				\item This instruction returns from an interrupt if
					\texttt{dA} is non-zero, setting \texttt{ie} to
					\texttt{0b1}.
				\item This instruction causes an exception if the processor
					is in user mode.
				%--------
				\end{itemize}

			\item \texttt{retx dA}
				\begin{itemize}
				%--------
				\item This instruction returns from supervisor mode to user
					mode if \texttt{dA} is non-zero.
				\item This instruction causes an exception if the processor
					is in user mode.
				%--------
				\end{itemize}
			\item \textit{Reserved for future expansion.}
			\item \textit{Reserved for future expansion.}
			\item \textit{Reserved for future expansion.}

			\item \textit{Reserved for future expansion.}
			\item \textit{Reserved for future expansion.}
			\item \textit{Reserved for future expansion.}
			\item \textit{Reserved for future expansion.}

			\item \textit{Reserved for future expansion.}
			\item \textit{Reserved for future expansion.}
			\item \textit{Reserved for future expansion.}
			\item \textit{Reserved for future expansion.}
			%--------
			\end{enumerate}
		%--------
		\end{itemize}
		\newpage

	\subsection{Group 9 Instructions}
		\begin{itemize}
		%--------
		\item Encoding: \texttt{1001 aaaa aabb bbbb  cccc ccii iiii Sooo}
			\begin{itemize}
			%--------
			\item \texttt{a}: ILAR a or DLAR a
			\item \texttt{b}: ILAR b or DLAR b
			\item \texttt{c}: DLAR c
			\item \texttt{i}: \texttt{imm6} (zero-extended 6-bit
				immediate), amount of consecutive LARs to use
			\item \texttt{S}:
				\begin{itemize}
				%--------
				\item when \texttt{0b0}: 
					Destination LARs (the ones starting with \texttt{iA} or
					\texttt{dA}) or source LARs (the ones starting with
					\texttt{iB}/\texttt{dB} and, if the instruction uses
					it, \texttt{dC}) are supervisor mode LARs.
					An example of the syntax for \texttt{getaddrs} in this
					case is \texttt{getaddrs.U}. The "\texttt{.U}" suffix
					indicates this instruction will have the \texttt{S} bit
					set to \texttt{0b0}.
				\item when \texttt{0b1}:
					Destination LARs (the ones starting with \texttt{iA} or
					\texttt{dA}) or source LARs (the ones starting with
					\texttt{iB}/\texttt{dB} and, if the instruction uses
					it, \texttt{dC}) are supervisor mode LARs.
					An example of the syntax for \texttt{getaddrs} in this
					case is \texttt{getaddrs.S}. The "\texttt{.S}" suffix
					indicates this instruction will have the \texttt{S} bit
					set to \texttt{0b1}.
				\item Note: whether \texttt{S} applies to destination LARs
					or source LARs is indicated in the description of the
					instruction.
				%--------
				\end{itemize}
			\item \texttt{o}: Opcode
			%--------
			\end{itemize}
		\item Instruction List:
			\begin{enumerate}
			%--------
			\item \texttt{getaddrs dA, dB, imm6}
				\begin{itemize}
				%--------
				\item This instruction uses the \texttt{S} bit to indicate
					which mode the source DLARs belong to.
				\item This instruction grabs the addresses of source DLARs
					starting with \texttt{dB} and then also the following
					\texttt{imm6 - 1} source DLARs. The grabbed addresses
					are then written into consecutive scalar data elements
					of destination DLARs (starting with \texttt{dA} and
					continuing into the following destination DLARs as
					necessary).
				\item When supervisor mode LARs are used for the source(s),
					this instruction causes an exception if used in user
					mode.
				%--------
				\end{itemize}
			\item \texttt{getaddrs dA, iB, imm6}
				\begin{itemize}
				%--------
				\item This instruction uses the \texttt{S} bit to indicate
					which mode the source ILARs belong to.
				\item This instruction grabs the addresses of source ILARs
					starting with \texttt{dB} and then also the following
					\texttt{imm6 - 1} source DLARs. The grabbed addresses
					are then written into consecutive scalar data elements
					of destination DLARs (starting with \texttt{dA} and
					continuing into the following destination DLARs as
					necessary).
				\item When supervisor mode LARs are used for the source(s),
					this instruction causes an exception if used in user
					mode.
				%--------
				\end{itemize}
			\item \texttt{gettypes dA, dB, imm6}
				\begin{itemize}
				%--------
				\item This instruction uses the \texttt{S} bit to indicate
					which mode the source DLARs belong to.
				\item This instruction grabs the types of source DLARs
					starting with \texttt{dB} and then also the following
					\texttt{imm6 - 1} source DLARs. The grabbed types are
					then written into consecutive scalar data elements of
					destination DLARs (starting with \texttt{dA} and
					continuing into the following destination DLARs as
					necessary).
				\item When supervisor mode LARs are used for the source(s),
					this instruction causes an exception if used in user
					mode.
				%--------
				\end{itemize}
			\item \texttt{ldm dA, dB, dC, imm6}
				\begin{itemize}
				%--------
				\item This instruction's name is short for "load multiple".
				\item This instruction uses the \texttt{S} bit to indicate
					to which mode the destination DLARs belong.
				\item This instruction uses addresses stored in the
					\texttt{imm6} scalar data elements of consecutive DLARs
					(starting with \texttt{dB}) and types stored in the
					\texttt{imm6} scalar data elements of consecutive DLARs
					(starting with \texttt{dC}). Multiple loads from memory
					are performed into the \texttt{imm6} destination DLARs
					(starting with \texttt{dA}).
				\item When supervisor mode LARs are used for the
					destination(s), this instruction causes an exception if
					used in user mode.
				%--------
				\end{itemize}

			\item \texttt{fetchm iA, dB, imm6}
				\begin{itemize}
				%--------
				\item This instruction's name is short for "fetch
					multiple".
				\item This instruction uses the \texttt{S} bit to indicate
					to which mode the destination DLARs belong.
				\item This instruction uses addresses stored in the
					\texttt{imm6} scalar data elements of consecutive DLARs
					(starting with \texttt{dB}). Multiple fetches from
					memory are performed into the \texttt{imm6} destination
					ILARs (starting with \texttt{iA}).
				\item When supervisor mode LARs are used for the
					destination(s), this instruction causes an exception if
					used in user mode.
				%--------
				\end{itemize}
			\item \texttt{reload dA, imm6}
				\begin{itemize}
				%--------
				\item This instruction forcibly re-loads the \texttt{imm6}
					consecutive DLARs, starting with \texttt{dA}, from
					memory. The \texttt{dirty} flags of each of these
					DLARs' shared data elements are cleared.
				%--------
				\end{itemize}
			\item \texttt{flush dA, imm6}
				\begin{itemize}
				%--------
				\item This instruction forcibly flushes the \texttt{imm6}
					consecutive DLARs, starting with \texttt{dA}, to
					memory. The \texttt{dirty} flags of each of these
					DLARs' shared data elements are cleared.
				%--------
				\end{itemize}
			\item \texttt{reload iA, imm6}
				\begin{itemize}
				%--------
				\item This instruction forcibly re-loads the \texttt{imm6}
					consecutive ILARs, starting with \texttt{iA}, from
					memory.
				\item Note: This instruction is potentially useful for
					self-modifying code.
				%--------
				\end{itemize}
			%--------
			\end{enumerate}
		%--------
		\end{itemize}
		\newpage

	\subsection{Group 10 Instructions}
		\begin{itemize}
		%--------
		\item Encoding: \texttt{1010 aaaa aabb bbbb  cccc ccdd dddd oooo}
			\begin{itemize}
			%--------
			\item \texttt{a}: DLAR a
			\item \texttt{b}: DLAR b
			\item \texttt{c}: DLAR c
			\item \texttt{d}: DLAR d
			\item \texttt{o}: Opcode
			%--------
			\end{itemize}
		\item Instruction List:
			\begin{enumerate}
			%--------
			\item \texttt{cpy.s dA, ie}
				\begin{itemize}
				%--------
				\item This instruction copies \texttt{ie} to
					\texttt{dA.scalar\_data}.
				%--------
				\end{itemize}
			\item \texttt{cpy.s ie, dA}
				\begin{itemize}
				%--------
				\item This instruction copies \texttt{dA.scalar\_data} to
					\texttt{ie}.
				%--------
				\end{itemize}
			\item \texttt{cpy.s dA, xct}
				\begin{itemize}
				%--------
				\item This instruction copies \texttt{xct} to
					\texttt{dA.scalar\_data}.
				%--------
				\end{itemize}
			\item \texttt{cpy.s xct, dA}
				\begin{itemize}
				%--------
				\item This instruction copies \texttt{dA.scalar\_data} to
					\texttt{xct}.
				%--------
				\end{itemize}

			\item \texttt{cpy.v dA, swiarg0}
				\begin{itemize}
				%--------
				\item This instruction copies \texttt{swiarg0} to
					\texttt{dA.vector\_data}.
				%--------
				\end{itemize}
			\item \texttt{cpy.v swiarg0, dA}
				\begin{itemize}
				%--------
				\item This instruction copies \texttt{dA.vector\_data} to
					\texttt{swiarg0}.
				%--------
				\end{itemize}
			\item \texttt{cpy.v dA, swiarg1}
				\begin{itemize}
				%--------
				\item This instruction copies \texttt{swiarg1} to
					\texttt{dA.vector\_data}.
				%--------
				\end{itemize}
			\item \texttt{cpy.v swiarg1, dA}
				\begin{itemize}
				%--------
				\item This instruction copies \texttt{dA.vector\_data} to
					\texttt{swiarg1}.
				%--------
				\end{itemize}

			\item \texttt{cpy.v dA, swiarg2}
				\begin{itemize}
				%--------
				\item This instruction copies \texttt{swiarg2} to
					\texttt{dA.vector\_data}.
				%--------
				\end{itemize}
			\item \texttt{cpy.v swiarg2, dA}
				\begin{itemize}
				%--------
				\item This instruction copies \texttt{dA.vector\_data} to
					\texttt{swiarg2}.
				%--------
				\end{itemize}
			\item \texttt{cpy.v dA, swiarg3}
				\begin{itemize}
				%--------
				\item This instruction copies \texttt{swiarg3} to
					\texttt{dA.vector\_data}.
				%--------
				\end{itemize}
			\item \texttt{cpy.v swiarg3, dA}
				\begin{itemize}
				%--------
				\item This instruction copies \texttt{dA.vector\_data} to
					\texttt{swiarg3}.
				%--------
				\end{itemize}

			\item \texttt{swi dA, dB, dC, dD}
				\begin{itemize}
				%--------
				\item Note that this instruction always causes an exception
					to occur.
				\item This instruction copies \texttt{dA.vector\_data} to
					to \texttt{swiarg0}.
				\item This instruction copies \texttt{dB.vector\_data} to
					to \texttt{swiarg1}.
				\item This instruction copies \texttt{dC.vector\_data} to
					to \texttt{swiarg2}.
				\item This instruction copies \texttt{dD.vector\_data} to
					to \texttt{swiarg3}.
				%--------
				\end{itemize}
			\item \textit{Reserved for future expansion.}
			\item \textit{Reserved for future expansion.}
			\item \textit{Reserved for future expansion.}
			%--------
			\end{enumerate}
		%--------
		\end{itemize}
		\newpage

	\subsection{Group 11 Instructions}
		\begin{itemize}
		%--------
		\item Encoding: \texttt{1011 aaaa aabb bbbb  cccc cc00 00vo oooo}
			\begin{itemize}
			%--------
			\item \texttt{a}: DLAR a
			\item \texttt{b}: DLAR b
			\item \texttt{c}: DLAR c
			\item \texttt{v}:
				\begin{itemize}
				%--------
				\item when \texttt{0b0}: scalar operation. The assembly
					syntax indicating a scalar operation simply adds
					"\texttt{.s}" to the instruction's name.
				\item when \texttt{0b1}: vector operation. The assembly
					syntax indicating a vector operation simply adds
					"\texttt{.v}" to the instruction's name.
				%--------
				\end{itemize}
			\item \texttt{o}: Opcode
			%--------
			\end{itemize}
		\item These instructions perform a read from/write to the IO
			address calculated by the following formula:
			\texttt{cast(u32, dB.scalar\_data)
				+ cast(u32, dC.scalar\_data)}.
		\item When a scalar operation is being performed, only the
			scalar data of \texttt{dA} is read into from/written out to IO
			space.
		\item When a vector operation is being performed, the
			entire vector data of \texttt{dA} is read into from/written out
			to IO space.
		\item Instruction List:
			\begin{enumerate}
			%--------
			\item \texttt{inu8 dA, dB, dC}
				\begin{itemize}
				%--------
				\item This instruction sets the type of \texttt{dA} to
					\texttt{u8} before performing anything else of the
					operation.
				\item This instruction reads from IO space and writes to
					\texttt{dA}.
				%--------
				\end{itemize}
			\item \texttt{ins8 dA, dB, dC}
				\begin{itemize}
				%--------
				\item This instruction sets the type of \texttt{dA} to
					\texttt{s8} before performing anything else of the
					operation.
				\item This instruction reads from IO space and writes to
					\texttt{dA}.
				%--------
				\end{itemize}
			\item \texttt{inu16 dA, dB, dC}
				\begin{itemize}
				%--------
				\item This instruction sets the type of \texttt{dA} to
					\texttt{u16} before performing anything else of the
					operation.
				\item This instruction reads from IO space and writes to
					\texttt{dA}.
				%--------
				\end{itemize}
			\item \texttt{ins16 dA, dB, dC}
				\begin{itemize}
				%--------
				\item This instruction sets the type of \texttt{dA} to
					\texttt{s16} before performing anything else of the
					operation.
				\item This instruction reads from IO space and writes to
					\texttt{dA}.
				%--------
				\end{itemize}

			\item \texttt{inu32 dA, dB, dC}
				\begin{itemize}
				%--------
				\item This instruction sets the type of \texttt{dA} to
					\texttt{u32} before performing anything else of the
					operation.
				\item This instruction reads from IO space and writes to
					\texttt{dA}.
				%--------
				\end{itemize}
			\item \texttt{ins32 dA, dB, dC}
				\begin{itemize}
				%--------
				\item This instruction sets the type of \texttt{dA} to
					\texttt{s32} before performing anything else of the
					operation.
				\item This instruction reads from IO space and writes to
					\texttt{dA}.
				%--------
				\end{itemize}
			\item \texttt{inu64 dA, dB, dC}
				\begin{itemize}
				%--------
				\item This instruction sets the type of \texttt{dA} to
					\texttt{u64} before performing anything else of the
					operation.
				\item This instruction reads from IO space and writes to
					\texttt{dA}.
				%--------
				\end{itemize}
			\item \texttt{ins64 dA, dB, dC}
				\begin{itemize}
				%--------
				\item This instruction sets the type of \texttt{dA} to
					\texttt{s64} before performing anything else of the
					operation.
				\item This instruction reads from IO space and writes to
					\texttt{dA}.
				%--------
				\end{itemize}

			%\item \texttt{inf32 dA, dB, dC}
			%	\begin{itemize}
			%	%--------
			%	\item This instruction sets the type of \texttt{dA} to
			%		\texttt{f32} before performing anything else of the
			%		operation.
			%	\item This instruction reads from IO space and writes to
			%		\texttt{dA}.
			%	%--------
			%	\end{itemize}
			%\item \texttt{inf64 dA, dB, dC}
			%	\begin{itemize}
			%	%--------
			%	\item This instruction sets the type of \texttt{dA} to
			%		\texttt{f64} before performing anything else of the
			%		operation.
			%	\item This instruction reads from IO space and writes to
			%		\texttt{dA}.
			%	%--------
			%	\end{itemize}
			\item \textit{Reserved for future expansion.}
			\item \textit{Reserved for future expansion.}
			\item \textit{Reserved for future expansion.}
			\item \textit{Reserved for future expansion.}

			\item \textit{Reserved for future expansion.}
			\item \textit{Reserved for future expansion.}
			\item \textit{Reserved for future expansion.}
			\item \textit{Reserved for future expansion.}
			%--------
			\item \texttt{outu8 dA, dB, dC}
				\begin{itemize}
				%--------
				\item This instruction sets the type of \texttt{dA} to
					\texttt{u8} before performing anything else of the
					operation.
				\item This instruction writes to IO space and reads from
					\texttt{dA}.
				%--------
				\end{itemize}
			\item \texttt{outs8 dA, dB, dC}
				\begin{itemize}
				%--------
				\item This instruction sets the type of \texttt{dA} to
					\texttt{s8} before performing anything else of the
					operation.
				\item This instruction writes to IO space and reads from
					\texttt{dA}.
				%--------
				\end{itemize}
			\item \texttt{outu16 dA, dB, dC}
				\begin{itemize}
				%--------
				\item This instruction sets the type of \texttt{dA} to
					\texttt{u16} before performing anything else of the
					operation.
				\item This instruction writes to IO space and reads from
					\texttt{dA}.
				%--------
				\end{itemize}
			\item \texttt{outs16 dA, dB, dC}
				\begin{itemize}
				%--------
				\item This instruction sets the type of \texttt{dA} to
					\texttt{s16} before performing anything else of the
					operation.
				\item This instruction writes to IO space and reads from
					\texttt{dA}.
				%--------
				\end{itemize}

			\item \texttt{outu32 dA, dB, dC}
				\begin{itemize}
				%--------
				\item This instruction sets the type of \texttt{dA} to
					\texttt{u32} before performing anything else of the
					operation.
				\item This instruction writes to IO space and reads from
					\texttt{dA}.
				%--------
				\end{itemize}
			\item \texttt{outs32 dA, dB, dC}
				\begin{itemize}
				%--------
				\item This instruction sets the type of \texttt{dA} to
					\texttt{s32} before performing anything else of the
					operation.
				\item This instruction writes to IO space and reads from
					\texttt{dA}.
				%--------
				\end{itemize}
			\item \texttt{outu64 dA, dB, dC}
				\begin{itemize}
				%--------
				\item This instruction sets the type of \texttt{dA} to
					\texttt{u64} before performing anything else of the
					operation.
				\item This instruction writes to IO space and reads from
					\texttt{dA}.
				%--------
				\end{itemize}
			\item \texttt{outs64 dA, dB, dC}
				\begin{itemize}
				%--------
				\item This instruction sets the type of \texttt{dA} to
					\texttt{s64} before performing anything else of the
					operation.
				\item This instruction writes to IO space and reads from
					\texttt{dA}.
				%--------
				\end{itemize}

			%\item \texttt{outf32 dA, dB, dC}
			%	\begin{itemize}
			%	%--------
			%	\item This instruction sets the type of \texttt{dA} to
			%		\texttt{f32} before performing anything else of the
			%		operation.
			%	\item This instruction writes to IO space and reads from
			%		\texttt{dA}.
			%	%--------
			%	\end{itemize}
			%\item \texttt{outf64 dA, dB, dC}
			%	\begin{itemize}
			%	%--------
			%	\item This instruction sets the type of \texttt{dA} to
			%		\texttt{f64} before performing anything else of the
			%		operation.
			%	\item This instruction writes to IO space and reads from
			%		\texttt{dA}.
			%	%--------
			%	\end{itemize}
			\item \textit{Reserved for future expansion.}
			\item \textit{Reserved for future expansion.}
			\item \textit{Reserved for future expansion.}
			\item \textit{Reserved for future expansion.}

			\item \textit{Reserved for future expansion.}
			\item \textit{Reserved for future expansion.}
			\item \textit{Reserved for future expansion.}
			\item \textit{Reserved for future expansion.}
			%--------
			\end{enumerate}
		%--------
		\end{itemize}
		\newpage

	%\subsection{Group 12 Instructions}
	%	\begin{itemize}
	%	%--------
	%	\item Encoding: \texttt{1100 aaaa aabb bbbb  iiii iiii iiii iiiv}
	%		\begin{itemize}
	%		%--------
	%		\item \texttt{a}: DLAR a
	%		\item \texttt{b}: DLAR b
	%		\item \texttt{i}: \texttt{simm15} (sign-extended 15-bit
	%			immediate)
	%		\item \texttt{v}:
	%			\begin{itemize}
	%			%--------
	%			\item when \texttt{0b0}: scalar operation. The assembly
	%				syntax indicating a scalar operation simply adds
	%				"\texttt{.s}" to the instruction's name.
	%			\item when \texttt{0b1}: vector operation. The assembly
	%				syntax indicating a vector operation simply adds
	%				"\texttt{.v}" to the instruction's name.
	%			%--------
	%			\end{itemize}
	%		%--------
	%		\end{itemize}
	%	\item Instruction List:
	%		\begin{enumerate}
	%		%--------
	%		\item \texttt{add dA, dB, ipc.addr, simm15}
	%		%--------
	%		\end{enumerate}
	%	%--------
	%	\end{itemize}
	%	\newpage

	%\subsection{Group 13 Instructions}
	%	\begin{itemize}
	%	%--------
	%	\item Encoding: \texttt{1101 aaaa aabb bbbb  iiii iiii iiii iooo}
	%		\begin{itemize}
	%		%--------
	%		\item \texttt{a}: DLAR a
	%		\item \texttt{b}: DLAR b
	%		\item \texttt{i}: \texttt{simm13} (sign-extended 13-bit
	%			immediate)
	%		\item \texttt{o}: Opcode
	%		%--------
	%		\end{itemize}
	%	\item For these instructions, the \texttt{dB} DLAR's scalar
	%		data field (temporarily casted to the \texttt{u32} type) and
	%		the \texttt{ipc} ILAR's address field are added together,
	%		plus \texttt{simm13} (sign-extended to 32-bit), to calculate
	%		the address being loaded from.
	%	\item Also, the type that \texttt{dA} is set to is indicated in the
	%		instruction name, with, for example, \texttt{ldu8} setting
	%		\texttt{dA}'s type to \texttt{u8}.
	%	\item Instruction List:
	%		\begin{enumerate}
	%		%--------
	%		\item \texttt{ldu8i dA, dB, ipc.addr, simm13}
	%		\item \texttt{lds8i dA, dB, ipc.addr, simm13}
	%		\item \texttt{ldu16i dA, dB, ipc.addr, simm13}
	%		\item \texttt{lds16i dA, dB, ipc.addr, simm13}
	%		\item \texttt{ldu32i dA, dB, ipc.addr, simm13}
	%		\item \texttt{lds32i dA, dB, ipc.addr, simm13}
	%		\item \texttt{ldu64i dA, dB, ipc.addr, simm13}
	%		\item \texttt{lds64i dA, dB, ipc.addr, simm13}
	%		%--------
	%		\end{enumerate}
	%	%--------
	%	\end{itemize}

	%\subsection{Group 14 Instructions}
	%	\begin{itemize}
	%	%--------
	%	\item Encoding: \texttt{1110 aaaa aa00 0000  0000 0000 iiii iioo}
	%		\begin{itemize}
	%		%--------
	%		\item \texttt{a}: DLAR a or ILAR a
	%		%\item \texttt{b}: DLAR b or ILAR b
	%		%\item \texttt{i}: \texttt{simm12} (sign-extended 12-bit
	%		%	immediate)
	%		\item \texttt{i}: \texttt{imm6} (zero-extended 6-bit immediate,
	%			the number of consecutive LARs following
	%			\texttt{dA} or \texttt{iA} to reload from/flush to memory)
	%		%--------
	%		\end{itemize}
	%	%\item Note that, for the first 10 instructions, memory is forcibly
	%	%	loaded from/stored to at the address calculated by the
	%	%	following formula:
	%	%	\texttt{cast(u32, dB.scalar\_data) + cast(s32, simm12)}.
	%	%	Also, for the first 10 instructions, if an address is to be
	%	%	evicted from the DLAR file, then it will be done before the
	%	%	data is forcibly loaded from memory.  Additionally, for the
	%	%	first 10 instructions, the \texttt{dirty} flag of the
	%	%	\texttt{dA} DLAR will be set to \texttt{0b0}, or
	%	%	\texttt{false}. Lastly, for these 10 instructions, the type
	%	%	that \texttt{dA} is set to is indicated by the name 
	%	\item Instruction List:
	%		\begin{enumerate}
	%		%--------
	%		%\item \texttt{fldu8i dA, dB, simm12}
	%		%\item \texttt{flds8i dA, dB, simm12}
	%		%\item \texttt{fldu16i dA, dB, simm12}
	%		%\item \texttt{flds16i dA, dB, simm12}

	%		%\item \texttt{fldu32i dA, dB, simm12}
	%		%\item \texttt{flds32i dA, dB, simm12}
	%		%\item \texttt{fldu64i dA, dB, simm12}
	%		%\item \texttt{flds64i dA, dB, simm12}
	%		%--------
	%		%\item \texttt{fldf32i dA, dB, simm12}
	%		%\item \texttt{fldf64i dA, dB, simm12}
	%		%\item \textit{Reserved for future expansion.}
	%		%\item \textit{Reserved for future expansion.}
	%		\item \texttt{reload dA, imm6}
	%			\begin{itemize}
	%			%--------
	%			\item This instruction forcibly re-loads the \texttt{imm6}
	%				consecutive DLARs, starting with \texttt{dA}, from
	%				memory. The \texttt{dirty} flags of each of these
	%				DLARs' shared data elements are cleared.
	%			%--------
	%			\end{itemize}
	%		\item \texttt{reload iA, imm6}
	%			\begin{itemize}
	%			%--------
	%			\item This instruction forcibly re-loads the \texttt{imm6}
	%				consecutive ILARs, starting with \texttt{iA}, from
	%				memory.
	%			%--------
	%			\end{itemize}
	%		\item \texttt{flush dA, imm6}
	%			\begin{itemize}
	%			%--------
	%			\item This instruction forcibly flushes the \texttt{imm6}
	%				consecutive DLARs, starting with \texttt{dA}, to
	%				memory. The \texttt{dirty} flags of each of these
	%				DLARs' shared data elements are cleared.
	%			%--------
	%			\end{itemize}
	%		\item \textit{Reserved for future expansion.}

	%		%\item \textit{Reserved for future expansion.}
	%		%\item \textit{Reserved for future expansion.}
	%		%\item \textit{Reserved for future expansion.}
	%		%\item \textit{Reserved for future expansion.}
	%		%--------
	%		\end{enumerate}
	%	%--------
	%	\end{itemize}


	%\printbibliography[heading=bibnumbered,title={Bibliography}]
%--------
\end{document}
