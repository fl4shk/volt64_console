\documentclass{article}

\usepackage{pbox}
\usepackage{graphicx}
\usepackage{float}
\usepackage{fancyvrb}
\usepackage[T1]{fontenc}
\usepackage{lmodern}
\usepackage{setspace}
\usepackage[nottoc]{tocbibind}
\usepackage[font=large]{caption}
\usepackage{framed}
\usepackage{tabularx}
\usepackage{amsmath}
\usepackage{hyperref}
\usepackage{fontspec}
\usepackage[backend=biber,sorting=none]{biblatex}
%%\usepackage[
%%	backend=biber,
%%	style=ieee,
%%	sorting=none
%%]{biblatex}
%\addbibresource{project_refs.bib}

%% Hide section, subsection, and subsubsection numbering
%\renewcommand{\thesection}{}
%\renewcommand{\thesubsection}{}
%\renewcommand{\thesubsubsection}{}

%% Alternative form of doing section stuff
%\renewcommand{\thesection}{}
%\renewcommand{\thesubsection}{\arabic{section}.\arabic{subsection}}
%\makeatletter
%\def\@seccntformat#1{\csname #1ignore\expandafter\endcsname\csname the#1\endcsname\quad}
%\let\sectionignore\@gobbletwo
%\let\latex@numberline\numberline
%\def\numberline#1{\if\relax#1\relax\else\latex@numberline{#1}\fi}
%\makeatother
%
%\makeatletter
%\renewcommand\tableofcontents{%
%    \@starttoc{toc}%
%}
%\makeatother

\newcommand{\respacing}{\doublespacing \singlespacing}
\newcommand{\code}[2][1]{\noindent \texttt{\tab[#1] #2} \\}
\newcommand{\skipline}[0]{\texttt{}\\}
\newcommand{\tnl}[0]{\tabularnewline}

\begin{document}
%--------
	\font\titlefont={Times New Roman} at 20pt
	\title{{\titlefont Example}}

	\font\bottomtextfont={Times New Roman} at 12pt
	\date{{\bottomtextfont} \today}
	\author{{\bottomtextfont Andrew Clark}}

	\setmainfont{Times New Roman}
	\setmonofont{Courier New}

	\maketitle
	\pagenumbering{gobble}

	\newpage
	\pagenumbering{arabic}
	%\tableofcontents
	%\newpage

	%\doublespacing

%\section{Abstract}
	%\setcounter{section}{-1}

\section{Table of Contents}
	\tableofcontents
	\newpage

\section{General Information}
	\begin{itemize}
	%--------
	\item The high level assembly language for \texttt{volt32\_cpu} is
		intended to be that of a memory-to-memory machine without the
		awkwardness of instruction LARs. Data LARs machines approximate
		memory-to-memory machines, and GCC (and probably LLVM) supports
		arbitrary addressing modes. The problem with attempting to use GCC
		for a DLARs machine is that multiple DLARs may update at the same
		time if they point to the same address, assuming the compiler
		doesn't know that they point to the same address.
	\item As a side note, there's an LLVM backend targeting the 6502, which
		basically uses the first 256 bytes of RAM as extra registers. Since
		the first 256 bytes of memory can be pointed to, this means that
		LLVM could probably deal with multiple DLARs that point to the same
		address updating at the same time.
	%--------
	\end{itemize}

\section{Instruction LARs}
	\begin{itemize}
	%--------
	\item "At some performance cost, it's pretty cheap to be excessive
		and always insert (\texttt{fetch}es) after branches and on
		alignment boundaries, or slightly clever and run a pessimistic
		heuristic cache policy in the assembler, since spurious loads
		(fetches) are almost free unless you evict something you
		needed."
	\item "I could see it being possible to handle ILARs like this:
		branches to places the assembler can't see (but perhaps the
		linker can) mean you don't know what ILARs have been modified

		and as such, when you return from a function not in the current
		translation unit, you `fetch` into a dedicated ILAR for
		following a branch

		well, for following a branch outside the translation unit

		You could handle pc-relative branches the same way, I think..."
	%--------
	\end{itemize}

\section{High Level, Pseudo Assembly Language}
	\begin{itemize}
	%--------
	\item Dedicate some number of DLARs (perhaps 32?) to act like regular
		registers. Call these "pseudo registers". These will be loaded, at
		the start of a function, with locations on the stack, close to the
		frame pointer. Only registers that get used by the function will be
		allocated on the stack. This enables us to have temporaries, backed
		up by memory. They will not be the \texttt{dA} argument of native
		loads and stores.
		\begin{itemize}
		%--------
		\item It should be possible to do spill/reload of these pseudo
			registers by just changing the address (load/store) of the real
			DLARs (by way of a native load/store).
		%--------
		\end{itemize}
	\item Include fake addresing modes that access "memory" at addresses
		calculated by \texttt{dsp + simm32}, \texttt{dfp + simm32},
		\texttt{pseudo\_register + simm32}, and \texttt{simm32}. These
		"memory" accesses will be preceded by native loads or stores into
		non-pseudo-register-DLARs that are managed entirely by the
		assembler. Allow generic three-argument instructions using these
		types of "memory" accesses, and/or pseudo registers, and/or
		immediates.
		\begin{itemize}
		%--------
		\item Optimize out spurious loads that I can prove are from
			addresses already in non-pseudo-reg DLARs. Use a combination of
			common subexpression elimiation and constant propagation.
		%--------
		\end{itemize}
	%--------
	\end{itemize}

	%\printbibliography[heading=bibnumbered,title={Bibliography}]

%--------
\end{document}
